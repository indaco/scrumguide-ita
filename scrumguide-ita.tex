%%%%%%%%%%%%%%%%%%%%%%%%%%%%%%%%%%%%%%%%%
% SCRUM Guide v. 2011 - main latex file %
%%%%%%%%%%%%%%%%%%%%%%%%%%%%%%%%%%%%%%%%%

\documentclass[a4paper,12pt,oneside,titlepage]{article}

\usepackage[plainpages=false]{hyperref}
\usepackage[italian]{babel}
\usepackage[utf8]{inputenc}
\usepackage[T1]{fontenc}
\usepackage{textcomp}
\usepackage{color}
\usepackage[layout=modern]{./layout/advancedcoverpage}
\usepackage[scaled=.98]{helvet}

% Header and Footer
\usepackage{fancyhdr}
\pagestyle{fancy}
\lhead{}
\chead{}
\rhead{}
\lfoot{© 1991-2011 Ken Schwaber and Jeff Sutherland, All Rights Reserved} 
\cfoot{}
\rfoot{| \thepage} 
\renewcommand{\headrulewidth}{0.4pt}
\renewcommand{\footrulewidth}{0.4pt}

% Scrum logo
\usepackage[cc]{./layout/titlepic}
\usepackage{graphicx}
\titlepic{\includegraphics[width=\textwidth]{./images/scrum_logo.jpg}}

\linespread{1.2}
\definecolor{Blue}{RGB}{63, 99, 141}
\definecolor{SteelBlue}{RGB}{70, 130, 180}

%%% Document
\begin{document}
	\date{}
	\title{}
	\maketitle
	\thispagestyle{fancy}
	\sffamily

	%%%%%%%%%%%%%%%%%%%%%%%%%%%%%%%
% SECTION 1 : Acknowledgement %
%%%%%%%%%%%%%%%%%%%%%%%%%%%%%%%
\section*{\color{Blue}{RINGRAZIAMENTI}}
\label{sec:acknowledgements}

\subsection*{\color{Blue}{GENERALI}}
\label{sec:general}
Scrum \`e basato su un insieme di prassi accettate nel settore industriale, utilizzate e collaudate da decenni. Si tratta di una teoria di processo con base empirica. Come disse Jim Coplien a Jeff: \flqq Scrum piacer\`a a chiunque; \`e ci\`o che gi\`a facciamo quando siamo con le spalle al muro.\frqq

\subsection*{\color{Blue}{PERSONE}}
\label{sec:people}
Delle migliaia di persone che hanno contribuito a Scrum, dovremmo individuare coloro che hanno contribuito nei suoi primi dieci anni. Prima c'erano Jeff Sutherland, in collaborazione con Jeff Mckenna, e Ken Schwaber con Mike Smith e Chris Martin. Scrum \`e stato formalmente presentato e pubblicato al OOPSLA del 1995. Nei 5 anni successivi, Mike Bandle e Martine Devos contribuirono in modo significativo. E per finire tutti gli altri, senza il loro aiuto non sarebbe stato possibile ridefinire Scrum in ci\`o che oggi \`e.

\subsection*{\color{Blue}{STORIA}}
\label{sec:history}
La storia di Scrum pu\`o gi\`a essere considerata lunga nel mondo dello sviluppo software.
Per ricordare i primi posti in cui \`e stato richiesto ed adottato, onoriamo Individual, Inc. Fidelity Investments, e IDX (oggi GE Medical).

%\section*{\color{Blue}{NOTE ALLA VERSIONE ITALIANA}}
%\label{sec:italianversion}
%TODO % Scopo della Guida a Scrum, Overview di Scrum
	%%%%%%%%%%%%%%%%%%%%%%%%%%%%%%%%%%%%%%%%%%%%%%%%%
% SECTION 2 : Purpose, Scrum Theory and pillars %
%%%%%%%%%%%%%%%%%%%%%%%%%%%%%%%%%%%%%%%%%%%%%%%%%
\newpage
\section*{\color{Blue}{LO SCOPO}}
\label{sec:purpose}
Scrum  \`e  stato  impiegato  per  sviluppare   prodotti
complessi sin dai  primi anni '90.  Questo documento descrive  come usare Scrum
per costruire prodotti.
Scrum non \`e un processo o una tecnica bens\`i un  framework all'interno del quale  possiamo utilizzare
vari processi e varie tecniche. Il ruolo di Scrum \`e quello di far emergere l'efficacia relativa
delle  pratiche di  sviluppo adottate,  in modo  da poterle migliorare, fornendo nel contempo un
framework adatto allo sviluppo di prodotti complessi.

\section*{\color{Blue}{LA TEORIA  DI SCRUM}}
\label{sec:scrum_theory}
Scrum si basa sulla teoria dei controlli empirici di processo. Utilizza un
metodo iterativo ed un  approccio incrementale per ottimizzare la prevedibilit\`a
ed  il  controllo  del  rischio.  Sono  tre  i  pilastri  che  sostengono   ogni
implementazione  del  controllo empirico  di  processo.

\subsection*{\color{Blue}{LA PRIMA  COLONNA  \`E LA TRASPARENZA}}
\label{sec:transparency}
La  trasparenza  garantisce  che   gli  aspetti  del  processo   che
influenzano il risultato siano visibili a coloro che gestiscono i risultati. Non
solo questi aspetti  devono essere trasparenti,  ma anche ci\`o  che si vede  deve
essere noto. Cio\`e, quando  qualcuno che ispeziona un processo crede che
qualcosa sia fatto, ci\`o deve equivalere alla sua definizione di fatto.

\subsection*{\color{Blue}{LA  SECONDA  COLONNA \`E  L'ISPEZIONE}}
\label{sec:inspection}
I  vari aspetti  del  processo devono essere controllati con una frequenza tale da permettere l'individuazione di  variazioni 
inaccettabili nello stesso. La frequenza delle ispezioni deve  prendere
in considerazione  il fatto  che tutti  i processi cambiano nel momento stesso in cui si esegue un'
ispezione. Un paradosso si verifica  quando la frequenza di controllo richiesta
supera  la  soglia tolleranza del processo all'ispezione.  Fortunatamente,  questo non
sembra  esser vero per lo sviluppo di software. L'altro  fattore sono l'abilit\`a e la
diligenza di chi ispeziona i risultati del lavoro.

\subsection*{\color{Blue}{LA TERZA  COLONNA \`E  L'ADATTAMENTO}}
Se chi  ispeziona verifica che uno o pi\`u aspetti del processo sono al di fuori dei limiti accettabili 
e che il prodotto finale non potr\`a essere accettato, deve regolare il processo o il
materiale lavorato.  La regolazione  deve essere  effettuata il  pi\`u rapidamente
possibile per ridurre al minimo l'ulteriore scarto.

In Scrum ci sono tre momenti dedicati all'ispezione e all'adattamento:
\begin{description}
\item[- l'incontro Daily  Scrum:]
	usato per controllare i progressi verso l'obiettivo dello Sprint e per procedere ad
	adattamenti  che  ottimizzano il  valore  del giorno  successivo  di lavoro.
\item[- gli incontri Sprint Review e  Planning Review:]
	usati per ispezionare i progressi  verso l'obiettivo di rilascio e per procedere con gli
	adattamenti che ottimizzano il valore del prossimo Sprint.
\item[- l'incontro Sprint Retrospective:]
	usato  per  esaminare il passato Sprint e determinare quali
	adattamenti potranno renderne il prossimo pi\`u produttivo, appagante  e
	divertente.
\end{description}
 % La Teoria di Scrum, Scrum
	%%%%%%%%%%%%%%%%%%%%%%%%%%%%%
% SECTION 3 : Scrum Content %
%%%%%%%%%%%%%%%%%%%%%%%%%%%%%
\section*{\color{Blue}{LA SOSTANZA DI SCRUM}}
Il framework Scrum  \`e costituito da una serie  di
\textbf{Scrum} Team e dai ruoli a essi associati; \textbf{Time-Box}, \textbf{Artefatti} e \textbf{Regole}.
I  Scrum Team sono concepiti per   ottimizzare la  flessibilit\`a e  la produttivit\`a. Proprio per questo
sono auto-organizzati, cross-funzionali e lavorano basandosi su iterazioni.
Ogni Scrum Team ha tre ruoli: 

\begin{enumerate}
\item \textbf{Scrum Master}: responsabile del fatto che il processo venga compreso e seguito;
\item \textbf{Product Owner}: responsabile di massimizzare il valore
del lavoro  che il  Team Scrum  fa; 
\item \textbf{Team}: fa  il lavoro. Il Team consiste di
sviluppatori con  tutte le  competenze necessarie  per tradurre le richieste  del
Product Owner in  una
porzione potenzialmente rilasciabile di prodotto entro  la fine  dello Sprint.
\end{enumerate}

Scrum impiega intervalli di tempo definiti (Time-Boxes) per creare regolarit\`a. Gli elementi di Scrum che sottostanno a
questa logica comprendono:
\begin{itemize}
\item \textbf{Release  Planning  Meeting} 
\item \textbf{Sprint Planning  Meeting}
\item \textbf{Sprint}
\item \textbf{Daily Scrum Meeting}
\item \textbf{Sprint Review}
\item \textbf{Sprint Retrospective}
\end{itemize}

Il cuore di Scrum \`e lo \textbf{Sprint}, cio\`e una iterazione di un mese o meno, di lunghezza coerente per
tutta la durata del progetto.

Tutti gli Sprint utilizzano  lo stesso framework Scrum  e tutti
gli Sprint  consegnano un  incremento del  prodotto finale  che \`e potenzialmente
rilasciabile.  Uno  Sprint inizia  immediatamente  dopo la fine del precedente.

Scrum  utilizza  quattro artefatti principali:\begin{itemize}
\item[-] \textbf{Product Backlog}: \`e l'elenco di tutto ci\`o che potrebbe essere necessario al prodotto, ordinato per
priorit\`a;
\item[-] \textbf{Sprint Backlog}: \`e la lista di compiti necessari a trasformare la parte di Product Backlog relativa a uno
Sprint in un incremento di prodotto potenzialmente rilasciabile;
\item[-] Burndown:  misura la quantit\`a residua di un Backlog nel corso del tempo;
\item[-] \textbf{Release Burndown}: misura la quantit\`a residua di Product Backlog rispetto al piano di rilascio (Release Plan);
\item[-] \textbf{Sprint Burndown}: misura la quantit\`a  di elementi residui di uno Sprint Backlog
nel corso  di uno  Sprint.
\end{itemize}

Vi sono delle \textbf{regole} che legano insieme i time-boxes, i ruoli e gli artefatti dello Scrum. Queste regole vengono
descritte nel presente documento. Per esempio, \`e regola dello Scrum che solo i membri del Team - le persone impegnate
a trasformare il Product Backlog in un incremento di prodotto - possono parlare durante un Daily Scrum Meeting. Le
modalit\`a di implementazione di Scrum che non sono regole, ma consigli, saranno presentati in riquadri specifici,
''consigli''.
\vspace{0.4cm}

\tip{Quando le regole non sono esplicitate, chi usa Scrum si aspetta di capire cosa fare. Non cercare di
trovare una soluzione perfetta, perch\`e le cose di solito cambiano rapidamente. Al contrario, prova qualcosa e vedi se
funziona. I meccanismi di natura empirica di Scrum, di ispezione-adeguamento, ti guideranno.} % Scrum Team
	%%%%%%%%%%%%%%%%%%%%%%%%%%%
% SECTION 4 : Scrum Roles %
%%%%%%%%%%%%%%%%%%%%%%%%%%%
\section*{\color{Blue}{I RUOLI IN SCRUM}}
\label{sec:roles}
Il Team Scrum \`e formato dallo Scrum Master, dal Product Owner e dal Team. I membri del team sono chiamati maiali
(pigs). Chiunque altro è un pollo (chicken). I ''polli'' non possono dire ai ''maiali'' come fare il loro lavoro.
La metafora di polli e maiali prende spunto dalla seguente storiella: 

\tale{"Ci sono un pollo e un maiale, e a un certo punto
il pollo dice: - Apriamo un ristorante! Il maiale ci pensa un po' su e poi chiede: - E come lo chiameremmo questo
ristorante? Il pollo risponde: - Uova e prosciutto! Al che il maiale dice: - No grazie, tu parteciperesti, ma solo io
sarei coinvolto seriamente!"}


\subsection*{\color{Blue}{LO SCRUM MASTER}}
\label{sec:scrummaster}
Lo Scrum Master \`e responsabile del fatto che il Team Scrum aderisca ai valori, alle pratiche e alle regole di Scrum.
Lo Scrum Master aiuta il Team e l'organizzazione in cui esso opera ad adottare Scrum. Lo Scrum Master insegna al Team,
e lo guida a essere pi\`u produttivo e ad aumentare la qualit\`a dei prodotti che sviluppa. Lo Scrum Master aiuta il
Team a comprendere e utilizzare i concetti di autoorganizzazione e cross-funzionalit\`a.\\ Lo Scrum Master inoltre aiuta il
Team a fare del proprio meglio in un ambiente organizzativo che pu\`o non essere ancora ottimizzato per lo sviluppo di
prodotti complessi. Quando lo Scrum Master contribuisce alla realizzazione di questi cambiamenti, parliamo di
eliminazione di ostacoli. 

\tip{Lo Scrum Master lavora assieme ai clienti e ai manager per identificare un Product Owner. Spiega al
Product Owner come fare il suo lavoro. Il Product Owner deve essere in grado di gestire l'ottimizzazione del valore del
prodotto usando Scrum. Se non lo riesce a fare, ne riteniamo responsabile lo Scrum Master.}

\tip{Lo Scrum Master pu\`o essere un membro del team, ad esempio, uno sviluppatore che ha dei task assegnati
all'interno dello Sprint. Tuttavia, questo spesso porta a conflitti quando lo Scrum Master si trova a dover scegliere
tra la rimozione di ostacoli e lo svolgimento di task assegnati a lui. Lo Scrum Master non dovrebbe essere mai il
Product Owner.}

\subsection*{\color{Blue}{IL PRODUCT OWNER}}
\label{sec:productowner}
Il Product Owner ha la responsabilit\`a, esclusivamente sua, di gestire il Product Backlog, e di garantire il valore
del lavoro svolto dal Team. Il Product Owner mantiene il Product Backlog e garantisce che sia visibile a tutti. Tutti
sanno quali sono gli elementi che hanno priorit\`a maggiore, e in questo modo tutti sanno su cosa si andr\`a, a
lavorare. \\ Il Product Owner \`e una persona, e non un comitato. Possono esserci dei comitati, che consigliano o
influenzano questa persona, ma chiunque voglia cambiare la priorit\`a di un elemento deve convincere il Product Owner.
Le aziende che adottano Scrum potranno notare con il tempo come esso influenza i modi in cui vengono stabilite
priorit\`a e requisiti.

\tip{Per lo sviluppo di prodotti commerciali, il Product Owner pu\`o essere il Product Manager. Per lo
sviluppo in-house, il Product Owner potrebbe essere il responsabile della funzione aziendale che viene automatizzata.}

\tip{Il Product Owner pu\`o essere un membro del Team, che fa nello stesso tempo anche lavoro di sviluppo.
Questa ulteriore responsabilit\`a pu\`o disturbare l'abilit\`a del Product Owner di lavorare con i vari stakeholder. In
ogni caso, il Product Owner non pu\`o mai essere lo Scrum Master.}

Affinch\`e il Product Owner abbia successo, all'interno dell'organizzazione tutti devono rispettare le sue decisioni. A
nessuno \`e permesso dire al Team di lavorare con un diverso ordine di priorit\`a, e i Team non sono autorizzati ad
ascoltare chi sostiene il contrario. Le decisioni del Product Owner sono visibili nel contenuto e nell'ordine delle
priorit\`a del Product Backlog. Questa visibilit\`a richiede al Product Owner di dare del proprio meglio, e rende il
suo ruolo estremamente impegnativo, ma anche molto gratificante .\\

\subsection*{\color{Blue}{IL TEAM}}
\label{sec:team}
I Team di sviluppatori trasformano i Product Backlog in incrementi di funzionalit\`a potenzialmente rilasciabili al
termine di ogni Sprint. I Team sono inoltre cross-funzionalit\`a; i membri di un Team devono avere tutte le competenze
necessarie per creare un incremento. I membri di un Team hanno spesso competenze specialistiche, quali programmazione,
controllo di qualit\`a, business analysis, architettura, design dell'interfaccia utente, o del database. Tuttavia, le
competenze che i membri del Team condividono ha - cio\`e, la capacit\`a di affrontare un requisito e trasformarlo in un
prodotto utilizzabile - tendono ad essere pi\`u importanti di quelle che non condividono. Le persone che si rifiutano
di scrivere codice perch\`e sono architetti o designer non si adattano bene a un Team. Tutti sono coinvolti, anche
quando questo richiede imparare cose nuove, o ricordarne di antiche. Non ci sono titoli in un Team, e non ci sono
eccezioni a questa regola. Inoltre, i Team non contengono sotto-team dedicati ad ambiti specifici, come ad esempio i
test o l'attivit\`a di analisi.\\ 
\linebreak In aggiunta, i Team sono auto-organizzati. Nessuno - neppure lo Scrum
Master - dice al Team come trasformare un elemento del Product Backlog in un incremento di funzionalit\`a. Il Team
gestisce questa cosa per conto proprio. Ogni membro del gruppo applica la propria esperienza a tutti i problemi
incontrati. La sinergia che ne risultata migliora l'efficacia e l'efficanza complessiva dell'intero Team.\\ 
\linebreak
La dimensione ottimale per un Team \`e di sette persone, pi\`u due, o meno due. Quando in un Team ci sono meno di
cinque membri, vi \`e una minore interazione, e ci\`o porta a un minore guadagno in termini di produttivit\`a. In
pi\`u, il Team pu\`o incontrare vincoli di competenze durante lo Sprint e pu\`o non essere in grado di consegnare una
parte rilasciabile del prodotto. Se invece ci sono pi\`u di nove membri, semplicemente il coordinamento risulta molto
oneroso. I Team grandi generano troppa complessit\`a, che non \`e possibile gestire con un processo empirico. Tuttavia,
abbiamo incontrato alcuni Team di successo che hanno superato il limite superiore o quello inferiore dell'intervallo
dimensionale suggerito. Il Product Owner e lo Scrum Master non sono inclusi nel conteggio, a meno che non siano anche
loro maiali.\\ 
\linebreak La composizione del Team pu\`o cambiare al termine di uno Sprint. Ogni qual volta un Team
cambia composizione, il livello di produttivit\`a raggiunto tramite l'auto-organizzazione diminuisce. Occorre prestare
particolare attenzione quando si cambia la composizione del Team. % Gli Eventi di Scrum
	%%%%%%%%%%%%%%%%%%%%%%%%%%
% SECTION 5 : Time-Boxes %
%%%%%%%%%%%%%%%%%%%%%%%%%%
\section*{\color{Blue}{TIME-BOXES}}
\label{sec:timeboxes}
Le Time-Boxes in Scrum sono il Release Planning Meeting, lo Sprint, lo Sprint Planning Meeting, lo Sprint Review, lo
Sprint Retrospective e il Daily Scrum.

\subsection*{\color{Blue}{RELEASE PLANNING MEETING}}
\label{sec:releaseplanningmeeting}
Lo scopo del Release Planning \`e quello di stabilire un piano e gli obiettivi che i team di Scrum e il resto delle
organizzazioni devono esser in grado di comprendere e comunicare. Il Release Planning  risponde alle domande: ''Qual \`e
il migliore modo possibile per trasformare la nostra visione in un prodotto vincente? Come possiamo raggiungere o superare la
soddisfazione desiderata del cliente e massimizzare il ritorno sull'investimento (ROI)?'' Il piano di rilascio stabilisce l'obiettivo
del rilascio, le priorit\`a pi\`u alte del Product Backlog, i rischi maggiori, le caratteristiche generali e le
funzionalit\`a che la versione da rilasciare conterr\`a. Stabilisce anche una probabile data di consegna e il costo che
dovrebbe contenere, se non cambia nulla. L'organizzazione pu\`o quindi controllare i progressi ed apportare modifiche a
questo piano Sprint-per-Sprint.
\newline

Il Release Planning Meeting \`e del tutto facoltativo. Se i Team Scrum iniziano i lavori senza la riunione,
l'assenza dei suoi artefatti sar\`a vista come un ostacolo da risolvere. Il lavoro per risolvere l'impedimento
diventer\`a un elemento del Product Backlog.
\newline

Con Scrum i prodotti sono costruiti in modo iterativo, in cui ogni Sprint crea un incremento del prodotto, a partire
dal pi\`u importante e pi\`u rischioso. Pi\`u Sprint creano pi\`u incrementi di prodotto. Ogni incremento \`e
potenzialmente rilasciabile. Quando ci sono abbastanza incrementi che danno valore al prodotto, questo viene
rilasciato. Molte organizzazioni hanno gi\`a un processo di pianificazione e di rilascio. Nella maggior parte di questi
processi la pianificazione \`e fatta all'inizio del rilascio per poi esser lasciata invariata con il passare del tempo.
In Scrum con il Release Planning Meeting vengono definiti l'obiettivo globale e i probabili esiti. Richiede solitamente non
pi\`u del 15-20\% del tempo necessario ad un'organizzazione per costruire un piano tradizionale rilascio. Tuttavia, un
rilascio gestito tramite Scrum realizza la pianificazione just-in-time ad ogni incontro di Sprint Review e di Sprint Planning ,
cos\`i come la pianificazione just-in-time quotidiana ad ogni riunione Daily Scrum. In generale, gli sforzi di rilascio
in Scrum probabilmente consumano un po' pi\`u di sforzo rispetto ad un piano tradizionale di rilascio.
\newline

Il Release Planning Meeting richiede la stima e la prioritizzazione del Product Backlog per il rilascio. Ci sono molte tecniche per
fare ci\`o, che si trovano al di fuori del terreno di competenza di Scrum, ma che sono comunque utili se usati con esso.

\subsection*{\color{Blue}{SPRINT}}
\label{sec:sprint}
Uno Sprint \`e una iterazione. Gli Sprint sono intervalli temporali. Durante lo Sprint, lo Scrum Master assicura che
non ci siano delle modifiche che alterino l'obiettivo di Sprint. Sia la composizione dei team che gli obiettivi di
qualit\`a rimangono costanti per tutta la sua durarta. Gli Sprint contengono e consistono della riunione Sprint
Planning, del lavoro di sviluppo, della Sprint Review e dello Sprint Retrospective. Gli Sprint si verificano uno dopo
l'altro, senza pause temporali tra essi.

\tip{Se un Team si accorge che si \`e impegnato a consegnare pi\`u di quanto pu\`o fare, il Team si incontra con il Product Owner per rimuovere o ridurre gli obiettivi del Product Backlog associati a quello Sprint. Se il Team crede invece di avere del tempo extra, pu\`o accordarsi con il Product Owner per selezionare altri elementi dal Product Backlog.}

Un progetto viene utilizzato per realizzare qualcosa; nel caso di sviluppo di software, \`e usato per costruire un
prodotto o un sistema. Ogni progetto consiste in una definizione di ci\`o che si va a costruire, un piano di
costruzione, il lavoro svolto in base al piano e il prodotto risultante. Ogni progetto ha un orizzonte temporale, vale
a dire i tempi per i quali il piano \`e da considerarsi buono. Se l'orizzonte \`e troppo lungo, la definizione potrebbe
essere cambiata, troppe variabili potrebbero essere entrate in gioco, il rischio pu\`o essere troppo grande ecc. Scrum \`e framework
per progetti il cui orizzonte non \`e pi\`u lungo di un mese, dove c'\`e complessit\`a sufficiente a far si che un
orizzonte temporale pi\`u lungo diventi troppo rischioso. La prevedibilit\`a del progetto deve essere controllata
almeno ogni mese, cos\`i il rischio che il progetto diventi imprevedibile e incontrollabile \`e contenuto ogni mese.

\tip{Quando un Team inizia con Scrum, Sprint di due settimane gli permettono di imparare senza brancolare nell'incertezza. Sprint di questa lunghezza possono essere sincronizzati con quelli di altri Team sommando due incrementi.}

Gli Sprint possono essere cancellati prima che la finestra di tempo dello Sprint si concluda. Solo il Product Owner ha
la facolt\`a di annullare la Sprint, anche se lui o lei pu\`o farlo anche sotto l'influenza degli stakeholder, del Team
o dello Scrum Master. Quali sono le circostanze per cui uno Sprint pu\`o essere cancellato? Potrebbe essere necessario
cancellare uno Sprint se l'obiettivo diventa obsoleto. In questo caso si tratta di una questione gestionale. Ci\`o
potrebbe verificarsi se la societ\`a cambia direzione o se cambiano le condizioni di mercato o della tecnologia. In
generale, uno Sprint deve essere annullato se, date le circostanze, non ha pi\`u senso. Tuttavia, a causa della breve
durata dell Sprint, raramente ha senso farlo.
\newline

Quando uno Sprint viene annullato, ogni elemento del Product Backlog completato e ''fatto'' viene esaminato. Sono
accettati solo gli elementi che rappresentano un incremento potenzialmente rilasciabile. Tutti gli altri elementi del Backlog
sono reimmessi nel Product Backlog con le loro stime iniziali. Qualsiasi lavoro svolto su di essi viene considerato
perduto. Terminare uno Sprint richiede consumo di risorse, dal momento che tutti vanno coinvolti in un altro incontro di
pianificazione affinch\`e si possa avviare un altro Sprint. Gli annullamenti di Sprint sono spesso traumatici per il
team, e sono molto rari.

\subsection*{\color{Blue}{SPRINT PLANNING MEETING}}
\label{sec:sprintplannnigmeeting}
Lo Sprint Planning Meeting \`e il momento in cui viene pianificata l'iterazione. Per uno Sprint della durata di un mese, questo
incontro ha la durata di otto ore. Per Sprint brevi, si pu\`o allocare circa il 5\% della lunghezza totale dello Sprint a questa
riunione. Lo Sprint Planning Meeting si compone di due parti. Nella prima parte, un tempo di quattro ore,  ci\`o che sar\`a fatto nel
Sprint \`e deciso. Nella seconda parte, le restanti quattro ore,  il Team parla di come andr\`a ad implementare
la funzionalit\`a tali da formare un incremento del prodotto durante lo Sprint.
\newline

Ci sono due parti dello Sprint Planning Meeting: ''Cosa?'' e ''Come?''. Alcuni Team Scrum combinano le due cose. Nella
prima parte, il Team Scrum affronta la questione del ''Cosa?'' Qui, il Product Owner presenta al team la parte del Product
Backlog con priorit\`a pi\`u alta. Product Owner e Team lavorano insieme per capire quali funzionalit\`a sviluppare nel corso del prossimo Sprint.
Gli input di questo incontro sono il Product Backlog, l'incremento pi\`u recente di prodotto, la capacit\`a del team ed
il suo rendimento passato. Solo il team in grado di valutare ci\`o che pu\`o compiere durante il prossimo Sprint.
\newline

Dopo aver selezionato  gli elementi dal Product Backlog, si decide l'obiettivo dello Sprint, che sar\`a raggiunto tramite
l'implementazione del Product Backlog. Questa \`e una dichiarazione che fornisce una bussola per il team che ha sempre
sotto occhio il perch\'e produrre un nuovo incremento. Lo Sprint Goal \`e un sottoinsieme dell'obiettivo di rilascio.
\newline

La ragione per avere un obiettivo di Sprint \`e quello di dare al team qualche spazio interpretativo per quanto
riguarda la funzionalit\`a. Ad esempio, l'obiettivo per lo Sprint di cui sopra potrebbe anche essere: ''Automatizzare
le funzionalit\`a di modifica dell'account cliente attraverso una possibilit\`a sicura e recuperabile di transazione a livello middleware''. Mentre il team lavora, mantiene questo obiettivo in mente. Al fine di soddisfare l'obiettivo, implementa
la funzionalit\`a e la tecnologia. Se il lavoro risulta essere pi\`u difficile di quanto il team si aspettava,
collabora con il Product Owner ed implementa solo parzialmente la funzionalit\`a. 
\newline

Nella seconda parte dello Sprint Planning Meeting, il Team affronta la questione del ''Come?'' Durante le seconde
quattro ore dallo Sprint Planning Meeting, il team calcola come trasformer\`a il Product Backlog selezionato durante
Sprint Planning Meeting (Cosa?) in un incremento di fatto. Il Team di solito inizia con la progettazione del lavoro.
Durante la progettazione, il team individua compiti. Questi compiti rappresentano i pezzi dettagliati dei lavori
necessari per convertire il Product Backlog in software. I compiti dovrebbero esser decomposti in modo tale da risultare
fattibili in meno di un giorno. Questo elenco si chiama Sprint Backlog. Il Team si auto-organizza e si impegna ad
assegnare i lavori dello Sprint Backlog, sia nel corso dello Sprint Planning Meeting sia durante lo Sprint. 
\newline

Il Product Owner \`e presente durante la seconda parte della riunione di Sprint Planning per chiarire il Product
Backlog e per contribuire a raggiungere dei compromessi. Se il Team valuta che ha troppo o troppo poco lavoro, si
pu\`o rinegoziare il Product Backlogcon il Product Owner. Il Team pu\`o anche invitare altre persone a partecipare al
fine di fornire una consulenza tecnica o su uno specifico dominio. Un nuovo team, durante questo incontro, spesso si rende subito
conto di dover affrontare il tutto come una squadra e non come unit\`a singola. Il Team si rende conto che deve fare
affidamento su se stesso. Quando si rende conto di ci\`o, comincia ad auto-organizzarsi per assumere le caratteristiche e il
comportamento di una vera squadra.

\tip{Di solito, solo il 60-70\% del Backlog totale dello Sprint viene strutturato nella riunione di Sprint Planning. La parte restante viene lasciata per essere dettagliata successivamente, o ne vengono fornite stime generiche, che verranno specificate meglio durante lo Sprint.}

\subsection*{\color{Blue}{SPRINT REVIEW}}
\label{sec:sprintreview}
Alla fine della Sprint si tiene l'incontro di Sprint Review. Quattro ore di riunione per uno Sprint di un mese. Nel
caso di Sprint di durata inferiore l'incontro non deve consumare pi\`u del 5\% del totale dello Sprint. Durante la Sprint Review, il
team di Scrum e le parti interessate discutono di ci\`o che \`e stato appena fatto. Sulla base di ci\`o, e sulle modifiche fatte al Product Backlog durante lo Sprint, si collabora per delineare cosa dovr\`a esser fatto il prossimo Sprint. Si
tratta di un incontro informale, con la presentazione della funzionalit\`a, destinato a promuovere la collaborazione su
cosa fare dopo. 
\newline

L'incontro comprende almeno i seguenti elementi. Il Product Owner identifica ci\`o che \`e stato fatto e ci\`o che non
\`e stato fatto. Il Team discute cosa \`e andato bene durante lo Sprint, quali problemi ha incontrato e come
risolverli. In pi\`u mostra il lavoro e risponde alle domande. Il Product Owner discute poi il Product
Backlog nella sua forma attuale. Lui o lei propone delle date di completamento con ipotesi di velocit\`a diverse.
L'intero gruppo collabora poi su ci\`o che ha visto e ci\`o che questo significa per le cose da fare dopo. La Sprint
Review fornisce un prezioso contributo alla successiva riunione di Sprint Planning.

\subsection*{\color{Blue}{SPRINT RETROSPECTIVE}}
\label{sec:sprintretrospective}
Dopo la Sprint Review e prima del prossimo incontro di Sprint Planning, il Team Scrum si riunisce per lo Sprint
Retrospective. In questa tre ore di riunione lo Scrum Master incoraggia il Team a rivedere, nel quadro del framework e
delle pratiche Scrum, il processo di sviluppo adottato al fine di rendere pi\`u efficace e piacevole il prossimo Sprint.
Molti libri documentano tecniche utili da usare durante la Sprint Review.
\newline

L'obiettivo della retrospezione (Sprint Review) \`e quello di esaminare come l'ultimo Sprint \`e andato per quanto riguarda le persone,
le relazioni, i processi e gli strumenti. L'ispezione dovrebbe individuare le priorit\`a e gli elementi principali che
sono andati bene e quegli elementi che, se affrontati in modo diverso, potrebbero rendere le cose migliori. Questi includono la
composizione del gruppo, le modalit\`a di riunione, gli strumenti, la definizione di ''fatto'', i metodi di
comunicazione, e i processi per trasformare gli elementi del Product Backlog in qualcosa di ''fatto.'' Entro la fine
dello Sprint Retrospective, il Team Scrum dovrebbe avere individuato le misure di miglioramenteo da attuare nei
prossimi Sprint. Questi cambiamenti diventano l'adattamento all'ispezione empirica.


\subsection*{\color{Blue}{DAILY SCRUM}}
\label{sec:dailyscrum}
Ogni team si incontra tutti i giorni per 15 minuti nel Daily Scrum. \`E un incontro che avviene sempre alla stessa ora
e sempre nello stesso luogo in cui ogni membro del team, spiega:
\begin{enumerate}
	\item ci\`o che  ha compiuto dopo l'ultima riunione;
	\item ci\`o che lui o lei si prepara a fare prima della prossima riunione; e
	\item quali ostacoli ci sono sulla sua strada.
\end{enumerate}

Il Daily Scrum migliora le comunicazioni, elimina le altre riunioni, individua e rimuove gli ostacoli allo sviluppo,
evidenzia e promuove un processo decisionale rapido ed accresce il livello di conoscenza del progetto di ciascuno.
\newline

Lo Scrum Master assicura che il team tanga la riunione. Il Team \`e responsabile della conduzione del Daily Scrum. Lo
Scrum Master insegna alla squadra a mantenere il Daily Scrum breve facendo rispettare le regole e facendo in modo che
la gente parli brevemente. Lo Scrum Master impone anche la regola che i polli non sono autorizzati a parlare n\`e a interferire in alcun
modo con il Daily Scrum.
 \newline

Il Daily Scrum non \`e una riunione di stato. Non \`e per chiunque, ma per chi pu\`o trasformare gli elementi del
Product Backlog in un incremento (il Team). Il Team si \`e impegnato per il raggiungimento dell'obiettivo di Sprint,  e per
gl elementi del Product Backlog scelti. Il Daily Scrum serve come controllo dei progressi che portano verso lo Sprint Goal (le
tre domande). 
Incontri successivi possono servire per fare adattamenti in relazione al lavoro previsto come prossimo all'interno dello Sprint.
L'intento \`e quello di ottimizzare la probabilit\`a che il team raggiunga il suo obiettivo. Nel processo
empirico alla base di Scrum il Daily Scrum rappresenta un incontro chiave di ispezione e adattamento.

 % Gli Artefatti di Scrum
	%%%%%%%%%%%%%%%%%%%%%%%%%%%%%%%
% SECTION 6 : Scrum Artifacts %
%%%%%%%%%%%%%%%%%%%%%%%%%%%%%%%
\section*{\color{Blue}{ARTEFATTI}}
\label{sec:artifacts}
Gli artefatti includono il Product Backlog, il Release Burndown, lo Sprint Backlog e lo Sprint Burndown.

\subsection*{\color{Blue}{PRODUCT BACKLOG E RELEASE BURNDOWN}}
\label{sec:productbacklog}
I requisiti per il prodotto che il Team (s) \`e in via di sviluppo sono elencati nella Backlog prodotto. Il
proprietario del prodotto \`e responsabile per l'arretrato del prodotto, il suo contenuto, la sua disponibilit\`a, e la
sua priorit\`a. Backlog prodotto non \`e mai completa. Il taglio iniziale di sviluppo che prevede solo la inizialmente
conosciuto e meglio comprendere i requisiti. Il portafoglio ordini del prodotto si evolve come il prodotto e l'ambiente
in cui verr\`a utilizzato evolve. L'arretrato \`e dinamico, nel senso che cambia continuamente di individuare ci\`o che
il prodotto deve essere adeguato, competitivo e utile. Finch\`e un prodotto esiste, esiste anche Backlog prodotto.\\
\linebreak 
Il portafoglio ordini del prodotto rappresenta tutto il necessario per sviluppare e lanciare un prodotto di
successo. Si tratta di un elenco di tutte le caratteristiche, le funzioni, le tecnologie, miglioramenti e correzioni di
bug, che costituiscono le modifiche che saranno apportate al prodotto per le versioni future. Oggetti Backlog Prodotto
hanno gli attributi di una descrizione, la priorit\`a, e stima. La priorit\`a \`e guidato da rischi, il valore, e la
necessit\`a. Ci sono molte tecniche per la valutazione di tali attributi.

\tip{da tradurre}

Backlog prodotto \`e ordinato secondo una priorit\`a. Backlog Top Product priorit\`a unit\`a le attivit\`a di sviluppo
immediato. Maggiore \`e la priorit\`a pi\`u urgente \`e, pi\`u si \`e pensato, e il consenso pi\`u c'\`e per quanto
riguarda il suo valore. Backlog priorit\`a pi\`u alt\`a \`e pi\`u chiaro e ha informazioni pi\`u dettagliate rispetto a
backlog priorit\`a pi\`u bassa. Meglio le stime sono fatte in base al maggior chiarezza e dettaglio maggiore. Pi\`u
bassa \`e la priorit\`a, tanto meno i dettagli, fino a quando non riesce a malapena a distinguere il prodotto.\\
\linebreak

Come un prodotto viene utilizzato, in quanto il suo valore aumenta, e come il mercato fornisce un feedback, arretrato
del prodotto emerge in una lista pi\`u ampia e completa. Requisiti non si fermano mai cambiare. Il Product Backlog \`e
un documento vivo. Cambiamenti nei requisiti di business, le condizioni di mercato, la tecnologia, e il personale
causare modifiche nel backlog prodotto. Per ridurre al minimo le rilavorazioni, solo gli elementi pi\`u alta priorit\`a
devono essere dettagliate fuori. Le voci Backlog prodotto che si occupano delle squadre per il prossimo sprint diversi
sono a grana fine, essendo stato scomposto in modo che ogni elemento pu\`o essere effettuata entro la durata della
Sprint.

\tip{da tradurre}

Multiple squadre Scrum spesso lavorano insieme su uno stesso prodotto. Un backlog di prodotto \`e utilizzato per
descrivere i lavori imminenti sul Prodotto. Un attributo Backlog prodotto che gli elementi dei gruppi viene poi
impiegato. Raggruppamento si pu\`o verificare dal set di funzionalit\`a, la tecnologia, o l'architettura, ed \`e spesso
usato come un modo per organizzare il lavoro con Scrum Team.

\tip{da tradurre}

Il grafico Release Burndown registra la somma di Portafoglio ordini residuo stimato lo sforzo prodotto nel tempo. Lo
sforzo \`e stimato in qualsiasi unit\`a di lavoro del Team Scrum e l'organizzazione sono decise. Le unit\`a di tempo
sono di solito Sprint.\\ 
\linebreak

Stime voce del Product Backlog sono calcolati inizialmente durante la Release Planning, e, successivamente, come
vengono creati. Durante Backlog prodotto grooming sono riesaminato e rivisto. Tuttavia, essi possono essere aggiornati
in qualsiasi momento. Il Team \`e responsabile di tutte le stime. Il Product Owner pu\`o influenzare la squadra,
aiutando a capire e selezionare trade-off, ma la stima finale \`e fatta dal team. Il Product Owner mantiene un elenco
aggiornato dei prodotti Backlog Backlog Release Burndown Posted in ogni momento. La linea di tendenza si possono trarre
sulla base del cambiamento di lavoro rimanente.

\tip{da tradurre}

\subsection*{\color{Blue}{SPRINT BACKLOG E SPRINT BURNDOWN}}
\label{sec:sprintbacklog}
Lo Sprint Backlog comprende i compiti del team esegue a sua volta elementi Backlog prodotto in un ''fatto'' di
incremento. Molti sono sviluppate durante la Sprint Planning Meeting. \`E tutto il lavoro che il team identifica come
necessario per soddisfare l'obiettivo Sprint. Oggetti Backlog Sprint deve essere scomposto. La decomposizione \`e
sufficiente modo che le modifiche in corso pu\`o essere compreso nel Daily Scrum.

\tip{da tradurre}

Il team di modifica Backlog Sprint tutta la Sprint, cos\`i come Backlog Sprint emergenti durante la Sprint. Come si
arriva in singole attivit\`a, si pu\`o scoprire che i compiti pi\`u o meno sono necessari, o che un determinato compito
sar\`a pi\`u o meno tempo di quanto fosse stato previsto. Come nuovo lavoro \`e necessario, il Team si aggiunge al
residuo Sprint. Come compiti sono manipolate o completati, le ore di lavoro stimato rimanente per ogni compito \`e
aggiornato. Quando le attivit\`a sono ritenute inutili, che vengono rimossi. Solo il team pu\`o cambiare il suo corso
di uno Sprint Backlog. Solo il team possono modificare i contenuti o le stime. Lo Sprint Backlog ha una grande
visibilit\`a, immagine in tempo reale del lavoro che il team ha in programma di realizzare nel corso della Sprint, ed
appartiene esclusivamente al team.\\ 
\linebreak

Backlog Sprint Burndown \`e un grafico della quantit\`a di lavoro Portafoglio ordini residuo in un Sprint Sprint
attraverso il tempo nella Sprint. Per creare questo grafico, determinare quanto lavoro resta sommando il ritardo stime
ogni giorno della Sprint. La quantit\`a di lavoro rimanente per un Sprint \`e la somma del lavoro rimanente per tutti
Backlog Sprint. Tenere traccia di queste somme di giorno e li usa per creare un grafico che mostra i restanti lavori
nel corso del tempo. Tracciando una linea attraverso i punti del grafico, il team in grado di gestire i suoi progressi
nel completamento di un lavoro di Sprint. Durata non \`e considerato in Scrum. Restanti lavori \`e la data sono le sole
variabili di interesse.\\ 
\linebreak 

Una delle regole Scrum appartiene alla fine di ogni Sprint, che \`e quello di fornire incrementi di funzionalit\`a
potenzialmente rilasciabile che aderisca ad una definizione operativa di ''fatto''.

\tip{da tradurre}

\subsection*{\color{Blue}{FATTO}}
\label{sec:done}
Scrum richiede squadre di costruire un incremento delle funzionalit\`a del prodotto ogni Sprint. Tale incremento deve
essere potenzialmente rilasciabile, per il prodotto proprietario pu\`o scegliere di applicare subito la funzionalit\`a.
Per fare ci\`o, l'incremento deve essere una fetta completa del prodotto. Essa deve essere ''fatto.'' Ogni incremento
dovrebbe essere a tutti gli additivi incrementi precedenti e testato, assicurando che tutti incrementi lavorare
insieme.

\tip{da tradurre}

Nello sviluppo del prodotto, affermando che la funzionalit\`a \`e fatto potrebbe portare qualcuno a ritenere che, in
almeno pulito codice, il refactoring, unit test, costruito e testato l'accettazione. Qualcun altro potrebbe supporre
solo che il codice \`e stato costruito. Se tutti non sa che la definizione di ''fatto'' \`e, gli altri due le gambe di
controllo del processo empirico non funzionano. Quando qualcuno descrive come fare qualcosa, ognuno deve capire cosa si
intende fare.\\
\linebreak

Fatto definisce ci\`o che il team intende quando si impegna a ''fare'' un elemento di backlog prodotto in uno sprint.
Alcuni prodotti non contengono la documentazione, quindi la definizione di ''fatto'' non comprende la documentazione. A
completamente ''fatto'' incremento include tutte le analisi, progettazione, refactoring, programmazione, documentazione
e analisi per l'incremento e tutti gli elementi backlog prodotto in incremento. Test comprende unit\`a, di sistema,
l'utente, e di regressione, così come non le prove funzionali, come le prestazioni, stabilit\`a, sicurezza e
integrazione. Fatto include qualsiasi internazionalizzazione. Alcune squadre non sono ancora in grado di comprendere
tutto il necessario per l'esecuzione nella loro definizione di fatto. Questo deve essere chiaro per il proprietario del
prodotto. Questo lavoro rimanente dovr\`a essere fatta prima che il prodotto pu\`o essere implementato e utilizzato. % Definizione di “Fatto”
	%%%%%%%%%%%%%%%%%%%%%%%%%%%%%%%%%%%%%%%%%%%%%%%%%%%%%%
% SECTION 7 : Conclusioni, Ringraziamenti, Revisioni %
%%%%%%%%%%%%%%%%%%%%%%%%%%%%%%%%%%%%%%%%%%%%%%%%%%%%%%
\newpage
\section*{\color{Blue}{Ringraziamenti}}
\label{sec:acknowledgements}

\subsection*{\color{Blue}{Persone}}
\label{sec:people}
Delle migliaia di persone che hanno contribuito a Scrum, dovremmo individuare coloro che hanno contribuito nei suoi
primi dieci anni. Prima c'erano Jeff Sutherland, in collaborazione con Jeff McKenna, e Ken Schwaber con Mike Smith e
Chris Martin. Molti altri hanno contribuito negli anni successivi e senza il loro aiuto non sarebbe stato possibile ridefinire Scrum in ci\`o che oggi \`e. David Starr ha approfondito e fornito le competenze editoriali per la formulazione di questa versione della Guida a Scrum.

\subsection*{\color{Blue}{Storia}}
\label{sec:history}
Ken Schwaber e Jeff Sutherland hanno presentato per la prima volta Scrum alla conferenza OOPSLA del 1995. Questa presentazione ha essenzialemente documentato ci\`o che Ken e Jeff avevano appreso negli anni precedenti applicando Scrum.

La storia di Scrum pu\`o gi\`a essere considerata lunga. Per ricordare i primi posti in cui \`e stato richiesto ed adottato, onoriamo Individual, Inc. Fidelity Investments, e IDX (oggi GE Medical).

\newpage
\section*{\color{Blue}{Revisioni}}
\label{sec:revisions}

Questa versione di luglio 2011 della Guida a Scrum è diversa dalla precedente di febbraio 2010. In particolare abbiamo cercato di rimuovere le tecniche o le best practices dal nucleo di Scrum. Questi variano a seconda della circostanza. Inizieremo pi\`u tardi con un compedio di ``Best Practices'' per offrire alcune delle nostre esperienze.

La Guida a Scrum Guida documenta come Scrum \`e stato sviluppato e sostenuto da pi\`u di venti anni da Jeff Sutherland e Ken Schwaber. Altre fonti forniscono modelli, processi e idee sulle pratiche, le agevolazioni e gli strumenti che completano il framewok Scrum. Queste ottimizzano la produttività, il valore, la creatività e l'orgoglio.

Le note di rilascio che coprono le seguenti differenze tra questa e la versione di Febbraio 2010 saranno pubblicate altrove, comprese le discussioni su:

\begin{enumerate}
 	\item Release Planning
	\item Release Burndown
	\item Sprint Backlog
	\item Product e Sprint Backlog Burndown
	\item Team (per Development Team)
	\item Maiali e Polli...la tradizione di Scrum
	\item Ordinati invece che prioritizzati 
\end{enumerate}

\newpage
\section*{\color{Blue}{Traduzione}}
\label{sec:translation}
Questo documento \`e stato tradotto dalla versione originale inglese di Ken Schwaber e Jeff Sutherland. Hanno contribuito a realizzare la traduzione Carlo Beschi e Mirco Veltri.

\section*{\color{Blue}{Note alla versione italiana}}
\label{sec:translationnotes}
I sorgenti della traduzione sono disponibili sul repository \htmladdnormallink{GitHub}{http://github.com/indaco/scrumguide-ita}. 
Sentitevi liberi di contattare i traduttori per segnalare errori o semplicemente per lasciare un feedback. Grazie.
 % Conclusioni, Ringraziamenti, Revisioni
\end{document}