%%%%%%%%%%%%%%%%%%%%%%%%%%%%%%%%%%%
% SECTION 4 : Gli Eventi di Scrum %
%%%%%%%%%%%%%%%%%%%%%%%%%%%%%%%%%%%

\section*{\color{Blue}{Gli Eventi di Scrum}}% (fold)
\label{sec:events}
\addcontentsline{toc}{section}{Gli Eventi di Scrum}
\pdfbookmark[10]{Gli Eventi di Scrum}{events}
Gli eventi previsti sono utilizzati in Scrum per creare regolarit\`a e ridurre al minimo la necessit\`a di riunioni non 
definite da Scrum stesso. Scrum utilizza eventi time-box, in modo che ogni evento ha una durata massima. Questo assicura che una 
quantit\`a appropriata di tempo è trascorsa pianificando senza permettere l'introduzione di sprechi nel processo di pianificazione. \newline
\\Oltre allo stesso Sprint, che \`e un contenitore di tutti gli altri eventi, ogni evento in Scrum \`e l'occasione per 
ispezionare ed adattare qualcosa. Questi eventi sono specificamente progettati per consentire trasparenza critica e ispezione. 
La mancata inclusione di uno qualsiasi dei risultati di questi eventi riduce la trasparenza ed \`e un'occasione persa per 
ispezionare e adattarsi.

\subsection*{\color{SteelBlue}{Lo Sprint}}% (fold)
\label{sec:sprint}
\addcontentsline{toc}{subsection}{Lo Sprint}
\pdfbookmark[11]{Lo Sprint}{sprint}

Il cuore di Scrum è uno Sprint, una time-box di un mese o meno durante il quale un ``Fatto'', utilizzabile, e un Incremento di  
prodotto potenzialmente rilasciabile \`e stato creato. Gli Sprint hanno una durata costante durante uno sforzo di sviluppo. Un 
nuova Sprint si avvia immediatamente dopo la conclusione della Sprint precedente. \newline
\\Lo Sprint contiene e consiste del meeting Sprint Planning, del Daily Scrum, del lavoro di sviluppo, del meeting Sprint Review 
e della Sprint Retrospective. \newline
\\Durante uno Sprint:

\begin{itemize}
	\item Non vengono apportate modifiche che potrebbero influenza l'obiettivo dello Sprint;
	\item La composizione del Team di Sviluppo e gli obiettivi di qualit\`a rimangono costanti; e,
	\item L'ambito pu\`o essere chiarito e rinegoziato tra il Product Owner ed il Team di Sviluppo quando si sar\`a imparato di 
	pi\`u.
\end{itemize}

Ogni Sprint pu\`o essere considerato un progetto con un orizzonte non pi\`u lungo di un mese. Come i progetti, gli Sprint sono 
utilizzabili per realizzare qualcosa. Ogni Sprint ha una definizione di ci\`o che si va a costruire, un progetto e un piano 
flessibile che guideranno la costruzione, il lavoro svolto e il prodotto risultante. \newline
\\Gli Sprint sono limitati ad un mese di calendario. Quando uno Sprint ha un orizzonte troppo lungo la definizione di ci\`o che 
viene costruito pu\`o cambiare, pu\`o aumentare la complessit\`a ed il rischio. Lo Sprint, almeno ogni mese solare, permette la 
prevedibilit\`a assicurando il controllo e l'dattamento del progresso verso un obiettivo. Lo Sprint limita anche il rischio di 
un mese di costo.

\subsubsection*{\color{SteelBlue}{Cancellare uno Sprint}}% (fold)
\label{sec:cancelling_sprint}
\pdfbookmark[11]{Cancellare uno Sprint}{cancelling_sprint}
Uno Sprint pu\`o essere cancellato prima che la finestra di tempo associata si concluda. Solo il Product Owner ha 
l'autorit\`a di annullare lo Sprint, anche se lui o lei pu\`o farlo anche sotto l'influenza degli stakeholder, del Team di 
Sviluppo o dello Scrum Master.\newline
\\Uno Sprint dovrebbe esser cancellato se l'Obiettivo di Sprint diventa obsoleto. Questo potrebbe verificarsi se la societ\`a 
cambia direzione o se cambiano le condizioni di mercato o della tecnologia. In generale, uno Sprint deve essere annullato se 
non ha pi\`u senso date le circostanze. Ma, data la sua breve durata, la cancellazione raramente ha senso. \newline
\\Quando uno Sprint viene annullato, ogni elemento del Product Backlog completato e ``Fatto'' viene esaminato. Se parti del 
lavoro sono state rilasciate, il Product Owner tipicamente le accetta. Tutte le voci incomplete del Product Backlog vengono ristimare e 
reimmesse nel Product Backlog. Il lavoro svolto su di esse si deprezza rapidamente e deve essere frequentemente ristimato. \newline
\\Le cancellazioni degli Sprint consumano risorse, dal momento che tutti vanno coinvolti in un altro Sprint Planning 
Meeting affinch\`e si possa avviare un altro Sprint. Le cancellazioni degli Sprint sono spesso traumatiche per il Team Scrum e 
sono molto rare.
% subsubsection cancelling_sprint (end)
% subsection sprint (end)

\subsection*{Sprint Plannig Meeting} % (fold)
\label{sub:sprint_plannig_meeting}
\addcontentsline{toc}{subsection}{Sprint Plannig Meeting}
\pdfbookmark[11]{Sprint Plannig Meeting}{sprint_plannig_meeting}
Il lavoro da svolgere  nello Sprint \`e pianificato durante lo Sprint Planning Meeting. Questo piano \`e creato dal lavoro 
collaborativo dell'intero Team Scrum. \newline
\\Il meeting Sprint Planning \`e un incontro della durata di otto ore per un Sprint di un mese. Per Sprint pi\`u brevi, 
l'evento \`e proporzionalmente pi\`u breve. Ad esempio, Sprint di due settimane hanno un meeting Sprint Planning di quattro 
ore. \newline
\\Il meeting Lo Sprint Planning si compone di due parti, ognuna delle quali della durata pari alla met\`a della durata del meeting 
Sprint Planning Meeting. Le due parti rispondono rispettivamente alle seguenti domande:

\begin{itemize}
	\item Cosa sar\`a rilasciato nell'Incremento derivante dal prossimo Sprint?
	\item Come sar\`a fatto il lavoro necessario a garantire il raggiungimento dell'Incremento?
\end{itemize}

\subsubsection*{Parte 1: Cosa sar\`a fatto in questo Sprint?} % (fold)
\label{ssub:part_one}
In questa parte, il Team di Sviluppo lavora per prevedere le funzionalit\`a che saranno sviluppate durante lo Sprint. Il 
Product Owner presenta le voci ordinate del Product Backlog al Team di Sviluppo e l'intero Team Scrum collabora sulla 
comprensione del lavoro dello Sprint. \newline
\\L'input per questo incontro sono il Product Backlog, l'ultimo Incremento prodotto, la capacit\`a di proiezione del Team di 
Sviluppo durante lo Sprint e le performance passate del Team di Sviluppo. Il numero di elementi selezionati dal Product Backlog 
per lo Sprint \`e definito dal Team di Sviluppo. Soltanto il Team di Sviluppo \`e in grado di valutare cosa pu\`o compiere 
durante il prossimo Sprint. \newline
\\Dopo che il Team di Sviluppo ha previsto le voci del Product Backlog che consegner\`a con lo Sprint, il Team Scrum definisce 
l'Obiettivo di Sprint. Si tratta dell'obiettivo che sar\`a raggiunto entro lo Sprint attraverso l'implementazione del Product 
Backlog e fornisce una guida per il Team di Sviluppo sul perch\`e si sta costruendo l'Incremento.
% subsubsection part_one (end)

\subsubsection*{Parte 2: Come si far\`a il lavoro scelto?} % (fold)
\label{ssub:parte_2_come_si_otterr`a_quello_che_`e_stato_scelto_}
Dopo aver selezionato il lavoro dello Sprint, il Team di Svilupo decide come costruir\`a, durante lo Sprint, questa 
funzionalit\`a in un Incremento ``Fatto'' del prodotto. Le voci del Product Backlog selezionate per lo Sprint pi\`u il piano 
per la consegna definiscono lo Sprint Backlog. \newline
\\Il Team di Sviluppo solitamente inizia con la progettazione del sistema e il lavoro necessario per convertire il Product 
Backlog in un Incremento del prodotto. Il lavoro pu\`o essere di varie dimensioni o sforzo stimato. Tuttavia, una quantit\`a 
sufficiente di lavoro \`e pianificata durante il meeting Sprint Planning per il Team di Sviluppo per prevedere ci\`o che ritiene 
di poter fare nel prossimo Sprint. Il lavoro pianificato per i primi giorni dello Sprint dal Team di Sviluppo viene suddiviso 
in unit\`a di un giorno o meno entro la fine di questo meeting. Il Team di Sviluppo si auto-organizza per intraprendere il 
lavoro contenuto nello Sprint Backlog, sia durante lo Sprint Planning meeting che quando necessario durante l'intero Sprint. \newline
\\Il Product Owner pu\`o essere presente durante la seconda parte dell'incontro di Sprint Planning per chiarire le voci 
selezionate dal Product Backlog e per contribuire a raggiungere dei compromessi. Se il Team di Sviluppo determina che c'\`e 
troppo o troppo poco lavoro, pu\`o rinegoziare le voci dello Sprint Backlog con il Product Owner. Il Team di Sviluppo pu\`o 
anche invitare altre persone a partecipare al meeting per una consulenza tecnica o di dominio. \newline
\\Alla fine dell'incontro di Sprint Planning, il Team di Sviluppo dovrebbe essere in grado di spiegare al Product Owner e allo 
Scrum Master come intende lavorare, in quanto team auto-organizzato, per raggiungere l'Obiettivo di Sprint e creare 
l'Incremento previsto.
% subsubsection part_2 (end)


\subsubsection*{Sprint Goal} % (fold)
\label{ssub:sprint_goal}
Lo Sprint Goal d\`a al Team di Sviluppo una certa flessibilit\`a per quanto riguarda le funzionalit\`a che saranno implementate 
durante lo Sprint. \newline
\\Mentre il Team di Sviluppo lavora, non perde mai di vista di questo obiettivo. Per soddisfare l'obiettivo, il Team implementa le 
funzionalit\`a e la tecnologia. Se il lavoro risulta essere diverso da quello previsto dal Team di Sviluppo allora i membri 
del Team collaborano con il Product Owner per negoziare l'ambito dello Sprint Backlog all'interno dello Sprint. \newline
\\L'Obiettivo di Sprint pu\`o essere considerato una milestone nella pi\`u ampia roadmap del prodotto.
% subsubsection sprint_goal (end)
% subsection sprint_plannig_meeting (end)

\subsection*{\color{SteelBlue}{Daily Scrum}}% (fold)
\label{sec:dailyscrum}
\addcontentsline{toc}{subsection}{Daily Scrum}
\pdfbookmark[13]{Daily Scrum}{dailyscrum}
L'incontro Daily Scrum \`e un evento della durata di 15 minuti che serve al Team di Sviluppo per sincronizzare le attivit\`a e creare un 
piano per le prossime 24 ore. Questo viene fatto controllando il lavoro svolto dopo l'ultimo Daily Scrum e la previsione del lavoro che si 
svolger\`a prima del prossimo meeting. \newline
\\Il Daily Scrum si svolge allo stesso orario e luogo ogni giorno per ridurre la complessit\`a. Durante l'incontro, ogni membro del team di 
Sviluppo spiega:

\begin{itemize}
    \item Che cosa è stato fatto dopo l'ultima riunione?
    \item Che cosa si farà prima della prossima riunione?
    \item Quali sono gli ostacoli incontrati?
\end{itemize}

Il Team di Sviluppo utilizza il Daily Scrum per valutare i progressi verso l'Obiettivo di Sprint e per valutare come il progresso \`e il 
linea verso il completamento del lavoro dello Sprint Backlog. Il Daily Scrum ottimizza la probabilit\`a che il Team di Sviluppo raggiunga 
l'Obiettivo di Sprint. Il Team di Sviluppo. Il Team di Sviluppo si incontra spesso subito dopo il Daily Scrum per ri-pianificare il resto 
del lavoro dello Sprint. Ogni giorno il Team di Sviluppo dovrebbe essere in grado di spiegare al Product Owner e allo Scrum Master come 
intende lavorare insieme come team auto-organizzatoz per raggiungere l'obiettivo e creare l'incremento atteso nel resto dello Sprint. \newline
\\Lo Scrum Master assicura che il Team di Sviluppo tenga la riunione ma il Team \`e responsabile della conduzione del Daily Scrum.  Lo
Scrum Master insegna alla squadra a mantenere il Daily Scrum della durata di 15 minuti. \newline
\\Lo Scrum Master impone la regola che soltanto i membri del Development Team possono partecipare al Daily Scrum. Questo incontro non \`e 
uno status meeting ed \`e rivolto alle persone che trasformano le voci del Product Backlog in un Incremento. \newline
\\ Il Daily Meeting migliora le comunicazioni, elimina altri incontri, identifica e rimuove gli ostacoli allo sviluppo, evidenzia e 
promuove il rapido processo decisionale e migliora il livello di conoscenza del progetto da parte del Team di Sviluppo. Rappresenta un 
incontro chiave di ispezione e adattamento.

% subsubsection dailyscrum (end)

\subsection*{\color{SteelBlue}{Sprint Review}}% (fold)
\label{sec:sprint_review}
\addcontentsline{toc}{subsection}{Sprint Review}
\pdfbookmark[14]{Sprint Review}{sprint_review}
Alla fine della Sprint si tiene la Sprint Review per ispezionare l'incremento e adattare, se necessario, il Product Backlog. 
Durante la riunione di Sprint Review il Team di Sviluppo e gli stakeholder collaborano su ci\`o che \`e stato fatto durante lo Sprint. 
Sulla base di questo e dei cambiamenti al Product Backlog fatti durante lo Sprint, i partecipanti collaborano sulle prossime cose che 
potrebbero esser fatte. Si tratta di un incontro informale e la presentazione dell'Incremento ha lo scopo di suscitare commenti e 
promuovere la collaborazione. \newline
\\Si tratta di un incontro della durata di quattro ore per uno Sprint di un mese. La durata \`e proporzionalmente inferiore per Sprint 
pi\`u brevi. Ad esempio, per uno Sprint di due settimane l'incontro di Sprint Review dura due ore. \newline
\\La Sprint Review include i seguenti elementi:

\begin{itemize}
	\item Il Product Owner identifica ci\`o che \`e stato ``Fatto'' e ci\`o che non \`e stato ``Fatto'';
	\item Il Team di Sviluppo discute su cosa \`e andato bene durante lo Sprint, quali problemi si sono incontrati e come questi problemi  
	sono stati risolti;
	\item Il Team di Sviluppo mostra il lavoro che ha ``Fatto'' e risponde alle domande sull'Incremento;
	\item Il Product Owner discute il Product Backlog cosi com'\`e. Lui o lei progetta la possibile data di completamento in base alla 
	misura del progresso fino ad oggi; e,
	\item L'intero gruppo collabora su cosa fare dopo, cos\`i la Sprint Review fornisce un prezioso contributo alle successive riunioni 
	di Sprint Planning.     
\end{itemize}

Il risultato della Sprint Review \`e  un Product Backlog rivisto che definisce quali sue probabili voci sono selezionabili per 
il prossimo Sprint. Il Product Backlog pu\`o anche essere modificato nel suo complesso per venire incontro a nuove opportunit\`a.

% subsubsection sprint_review (end)

\subsection*{\color{SteelBlue}{Sprint Retrospective}}% (fold)
\label{sec:sprint_retrospective}
\addcontentsline{toc}{subsection}{Sprint Retrospective}
\pdfbookmark[15]{Sprint Retrospective}{sprint_retrospective}
La Sprint Retrospective \`e l'occasione per il Team Scrum per ispezionare se stesso e creare un piano di miglioramento da attuare durante 
il prossimo Sprint. Dopo la Sprint Review e prima del prossimo incontro di Sprint Planning, il Team Scrum si riunisce per lo Sprint
Retrospective. Si tratta di una riunione di tre ore, per Sprint della durata mensile. In modo proporzionale allocate meno tempo per Sprint 
pi\`u brevi. \newline
\\Lo scopo della Sprint Retrospective \`e quello di:

\begin{itemize}
	\item Esaminare come l'ultimo Sprint \`e andato per quanto riguarda le persone, le relazioni, i processi e gli strumenti;
	\item Identificare ed ordinare i maggiori elementi che son andati bene ed il potenziale di miglioramento; e,
	\item Creare un piano per attuare i miglioramenti al modo di lavorare del Team Scrum.
\end{itemize}

\noindent Lo Scrum Master incoraggia il Team Scrum a migliorare, all'interno del framework di processo Scrum, il proprio processo di 
sviluppo e le pratiche per rendere pi\`u efficace e divertente il prossimo Sprint. Durante ogni Sprint Retrospective, il Team Scrum 
pianifica i modi per aumentare la qualit\`a del prodotto adattando la definizione di ``Fatto'' a seconda dei casi. \newline
\\Entro la fine della Sprint Retrospective, il Team Scrum dovrebbe aver individuato i miglioramenti che saranno implementati nel prossimo 
Sprint. Attuare tali miglioramenti durante il prossimo Sprint \`e l'adattamento all'ispezione del Team Scrum stesso. Anche se i 
miglioramenti possono essere implementati in ogni momento, la Sprint Retrospective fornisce un evento dedicato focalizzato sull'ispezione e 
l'adattamento.
% subsubsection sprint_retrospective (end)