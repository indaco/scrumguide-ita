%%%%%%%%%%%%%%%%%%%%%%%%%%%
% SECTION 4 : Scrum Roles %
%%%%%%%%%%%%%%%%%%%%%%%%%%%
\section*{\color{Blue}{I RUOLI IN SCRUM}}
\label{sec:roles}
Il Team Scrum \`e  formato dallo Scrum Master, dal Product Owner e dal Team.

\subsection*{\color{Blue}{LO SCRUM MASTER}}
\label{sec:scrummaster}
Lo Scrum Master \`e responsabile di assicurare che il Team Scrum aderisce ai valori, alle prassi e alle regole di Scrum. Lo Scrum Master aiuta il Team ad adottare Scrum; fa coaching al Team per consentirgli di essere pi\`u produttivo e ad aumentare la qualit\`a di ci\`o che sviluppa; aiuta il Team a comprendere e utilizzare i concetti di autogestione e cross-funzionalit\`a.\\
Lo Scrum Master aiuta ancora il Team a fare del proprio meglio in un ambiente organizzativo che non pu\`o ancora essere ottimizzato per lo sviluppo di prodotti complessi. Quando lo Scrum Master contribuisce a rendere attivi questi cambiamenti si parla di eliminazione di ostacoli. Ricordiamo per\`o che lo Scrum Master non gestisce il Team (\`e autoorganizzato).

\vspace{0.4cm}
\fbox{
\begin{minipage}{0.9\textwidth}
\begin{large}TIP\end{large}\\
Lo Scrum Master lavora con i clienti per identificare e creare un nuovo prodotto. Fa coaching al Product Owner su  come fare il suo lavoro con Scrum. I Product Owner devono essere informati su come riuscire ad ottimizzare il valore del prodotto usando Scrum. Se non lo fanno, la responsabilit\`a di Scrum Master sar\`a nostra.
\end{minipage}
}

\vspace{0.4cm}
\fbox{
\begin{minipage}{0.9\textwidth}
\begin{large}TIP\end{large}\\
Lo Scrum Master pu\`o essere un membro del team, ad esempio, uno sviluppatore. 
Tuttavia, questo spesso porta a conflitti quando lo Scrum Master deve scegliere tra la rimozione 
di ostacoli e lo svolgere dei task. Lo Scrum Master non dovrebbe mai essere il Product Owner.
\end{minipage}
}


\subsection*{\color{Blue}{IL PRODUCT OWNER}}
\label{sec:productowner}
Il Product Owner \`e il primo ed unico responsabile della gestione del \foreignlanguage{english}{Pro\-duct Backlog} e garantisce il valore dei lavori svolti dal Team. Questa persona sostiene gli arretrati del prodotto e garantisce che siano visibile a tutti. Tutti conoscono gli oggetti a massima priorit\`a, in modo che tutti conoscono ci\`o su cosa si lavora.\\
Il Product Owner \`e  una persona, non un comitato. I comitati possono esistere che consigliare o influenzare questa persona, ma chi vuole cambiare la priorit\`e di un elemento deve convincere il Product Owner.

\vspace{0.4cm}
\fbox{
\begin{minipage}{0.9\textwidth}
\begin{large}TIP\end{large}\\
Per lo sviluppo commerciale, il Product Owner pu\`o essere il Product Manager. Per lo sviluppo in-house, il Product Owner potrebbe essere il responsabile della funzione aziendale che viene automatizzata.
\end{minipage}
}


\vspace{0.4cm}
\fbox{
\begin{minipage}{0.9\textwidth}
\begin{large}TIP\end{large}\\
Il Product Owner pu\`o essere un membro del team, anche facendo il lavoro di sviluppo. Questa ulteriore responsabilit\`a pu\`o disturbare l'abilit\`a del Product Owner di lavorare con le parti interessate. Tuttavia, il Product Owner non pu\`o mai essere lo Scrum Master.
\end{minipage}
}
\vspace{0.4cm}
\linebreak
Il Product Owner per avere successo, deve veder rispettate le sue decisioni da parte di tutti i membri dell'organizzazione. A nessuno \`e  permesso dire al team di lavorare con un diverso insieme di priorit\`a, e i team non sono autorizzati ad ascoltare chi dice il contrario. Le decisioni del\foreignlanguage{english}{Product Backlog} sono visibili nel contenuto e nel dare una priorit\`a al ritardo del prodotto. Questa visibilit\`a richiede al \foreignlanguage{english}{Product Backlog} il meglio delle sue possibilit\`a e rende riconoscibile e gratificante il suo ruolo sul mercato.\\

\subsection*{\color{Blue}{IL TEAM}}
\label{sec:team}
Il Team di sviluppatori si occupa di trasformare Product Backlog in incrementi di funzionalit\`a potenzialmente rilasciabili ad ogni Sprint. I Team sono anche cross-funzionalit\`a; i membri del team devono avere tutte le competenze necessarie per creare un incremento del lavoro. I membri del team hanno spesso le competenze specialistiche, quali la programmazione, analisi di business, il controllo di qualit\`a, l'architettura, il design dell'interfaccia utente, o dati di progettazione di base. Tuttavia, le competenze che il membro della squadra ha - cio\`e, la capacit\`a di affrontare un requisito e trasformarlo in un prodotto utilizzabile - tendono ad essere pi\`u importanti di quelle che non ha. Le persone che si rifiutano di scrivere codice perch\`e sono architetti o designer, non si adattano bene ai Team. Tutti sono coinvolti, anche se questo richiede l'apprendimento di nuove competenze o il dover ricordare quelle precedenti. Non ci sono titoli nel Team, e non ci sono eccezioni a questa regola. I Team non si suddividono in mini-team dedicati a settori particolari, come i test o l'attivit\`a di analisi.\\
\linebreak
I Team sono anche auto-organizzati. Nessuno - nemmeno lo Scrum Master - dice al Team come trasformare gli elementi del Product Backlog in incrementi di funzionalit\`a. Il Team decide questo per proprio conto. Ogni membro del gruppo applica la sua esperienza a tutti i problemi. La sinergia dei risultati migliora l'efficacia complessiva di tutto il Team.\\
\linebreak
La dimensione ottimale per un Team \`e di sette persone, pi\`u o meno due. Quando ci sono meno di cinque membri del Team, vi \`e una minore interazione che causa un rallentamento dell'incremento di produttivit\`a. In pi\`u, il Team pu\`o incontrare vincoli di competenze durante lo Sprint e non essere in grado di consegnare parte del prodotto. Se ci sono pi\`u di nove membri il semplice coordinamento risulta oneroso. Grandi Team generano troppa complessit\`a per la gestione di un processo empirico. Tuttavia, abbiamo incontrato alcuni Team di successo che hanno superato i limiti superiori e inferiori di questo intervallo di grandezza. Il Product Owner e lo Scrum Master non sono inclusi nel conteggio.\\
\linebreak
La composizione del Team pu\`o cambiare al termine di uno Sprint. Ogni qual volta un Team cambia composizione, la  produttivit\`a raggiunta dall'auto-organizzazione diminuisce. Occorre prestare particolare attenzione quando si cambia la composizione del Team. 