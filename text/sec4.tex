%%%%%%%%%%%%%%%%%%%%%%%%%%%
% SECTION 4 : Scrum Roles %
%%%%%%%%%%%%%%%%%%%%%%%%%%%
\section*{\color{Blue}{I RUOLI IN SCRUM}}
\label{sec:roles}
Il Team Scrum \`e formato dallo Scrum Master, dal Product Owner e dal Team. I membri del team sono chiamati maiali
(pigs). Chiunque altro è un pollo (chicken). I ''polli'' non possono dire ai ''maiali'' come fare il loro lavoro.
La metafora di polli e maiali prende spunto dalla seguente storiella: 

\tale{"Ci sono un pollo e un maiale, e a un certo punto
il pollo dice: - Apriamo un ristorante! Il maiale ci pensa un po' su e poi chiede: - E come lo chiameremmo questo
ristorante? Il pollo risponde: - Uova e prosciutto! Al che il maiale dice: - No grazie, tu parteciperesti, ma solo io
sarei coinvolto seriamente!"}


\subsection*{\color{Blue}{LO SCRUM MASTER}}
\label{sec:scrummaster}
Lo Scrum Master \`e responsabile del fatto che il Team Scrum aderisca ai valori, alle pratiche e alle regole di Scrum.
Lo Scrum Master aiuta il Team e l'organizzazione in cui esso opera ad adottare Scrum. Lo Scrum Master insegna al Team,
e lo guida a essere pi\`u produttivo e ad aumentare la qualit\`a dei prodotti che sviluppa. Lo Scrum Master aiuta il
Team a comprendere e utilizzare i concetti di autoorganizzazione e cross-funzionalit\`a.\\ Lo Scrum Master inoltre aiuta il
Team a fare del proprio meglio in un ambiente organizzativo che pu\`o non essere ancora ottimizzato per lo sviluppo di
prodotti complessi. Quando lo Scrum Master contribuisce alla realizzazione di questi cambiamenti, parliamo di
eliminazione di ostacoli. 

\tip{Lo Scrum Master lavora assieme ai clienti e ai manager per identificare un Product Owner. Spiega al
Product Owner come fare il suo lavoro. Il Product Owner deve essere in grado di gestire l'ottimizzazione del valore del
prodotto usando Scrum. Se non lo riesce a fare, ne riteniamo responsabile lo Scrum Master.}

\tip{Lo Scrum Master pu\`o essere un membro del team, ad esempio, uno sviluppatore che ha dei task assegnati
all'interno dello Sprint. Tuttavia, questo spesso porta a conflitti quando lo Scrum Master si trova a dover scegliere
tra la rimozione di ostacoli e lo svolgimento di task assegnati a lui. Lo Scrum Master non dovrebbe essere mai il
Product Owner.}

\subsection*{\color{Blue}{IL PRODUCT OWNER}}
\label{sec:productowner}
Il Product Owner ha la responsabilit\`a, esclusivamente sua, di gestire il Product Backlog, e di garantire il valore
del lavoro svolto dal Team. Il Product Owner mantiene il Product Backlog e garantisce che sia visibile a tutti. Tutti
sanno quali sono gli elementi che hanno priorit\`a maggiore, e in questo modo tutti sanno su cosa si andr\`a, a
lavorare. \\ Il Product Owner \`e una persona, e non un comitato. Possono esserci dei comitati, che consigliano o
influenzano questa persona, ma chiunque voglia cambiare la priorit\`a di un elemento deve convincere il Product Owner.
Le aziende che adottano Scrum potranno notare con il tempo come esso influenza i modi in cui vengono stabilite
priorit\`a e requisiti.

\tip{Per lo sviluppo di prodotti commerciali, il Product Owner pu\`o essere il Product Manager. Per lo
sviluppo in-house, il Product Owner potrebbe essere il responsabile della funzione aziendale che viene automatizzata.}

\tip{Il Product Owner pu\`o essere un membro del Team, che fa nello stesso tempo anche lavoro di sviluppo.
Questa ulteriore responsabilit\`a pu\`o disturbare l'abilit\`a del Product Owner di lavorare con i vari stakeholder. In
ogni caso, il Product Owner non pu\`o mai essere lo Scrum Master.}

Affinch\`e il Product Owner abbia successo, all'interno dell'organizzazione tutti devono rispettare le sue decisioni. A
nessuno \`e permesso dire al Team di lavorare con un diverso ordine di priorit\`a, e i Team non sono autorizzati ad
ascoltare chi sostiene il contrario. Le decisioni del Product Owner sono visibili nel contenuto e nell'ordine delle
priorit\`a del Product Backlog. Questa visibilit\`a richiede al Product Owner di dare del proprio meglio, e rende il
suo ruolo estremamente impegnativo, ma anche molto gratificante .\\

\subsection*{\color{Blue}{IL TEAM}}
\label{sec:team}
I Team di sviluppatori trasformano i Product Backlog in incrementi di funzionalit\`a potenzialmente rilasciabili al
termine di ogni Sprint. I Team sono inoltre cross-funzionalit\`a; i membri di un Team devono avere tutte le competenze
necessarie per creare un incremento. I membri di un Team hanno spesso competenze specialistiche, quali programmazione,
controllo di qualit\`a, business analysis, architettura, design dell'interfaccia utente, o del database. Tuttavia, le
competenze che i membri del Team condividono ha - cio\`e, la capacit\`a di affrontare un requisito e trasformarlo in un
prodotto utilizzabile - tendono ad essere pi\`u importanti di quelle che non condividono. Le persone che si rifiutano
di scrivere codice perch\`e sono architetti o designer non si adattano bene a un Team. Tutti sono coinvolti, anche
quando questo richiede imparare cose nuove, o ricordarne di antiche. Non ci sono titoli in un Team, e non ci sono
eccezioni a questa regola. Inoltre, i Team non contengono sotto-team dedicati ad ambiti specifici, come ad esempio i
test o l'attivit\`a di analisi.\\ 
\linebreak In aggiunta, i Team sono auto-organizzati. Nessuno - neppure lo Scrum
Master - dice al Team come trasformare un elemento del Product Backlog in un incremento di funzionalit\`a. Il Team
gestisce questa cosa per conto proprio. Ogni membro del gruppo applica la propria esperienza a tutti i problemi
incontrati. La sinergia che ne risultata migliora l'efficacia e l'efficanza complessiva dell'intero Team.\\ 
\linebreak
La dimensione ottimale per un Team \`e di sette persone, pi\`u due, o meno due. Quando in un Team ci sono meno di
cinque membri, vi \`e una minore interazione, e ci\`o porta a un minore guadagno in termini di produttivit\`a. In
pi\`u, il Team pu\`o incontrare vincoli di competenze durante lo Sprint e pu\`o non essere in grado di consegnare una
parte rilasciabile del prodotto. Se invece ci sono pi\`u di nove membri, semplicemente il coordinamento risulta molto
oneroso. I Team grandi generano troppa complessit\`a, che non \`e possibile gestire con un processo empirico. Tuttavia,
abbiamo incontrato alcuni Team di successo che hanno superato il limite superiore o quello inferiore dell'intervallo
dimensionale suggerito. Il Product Owner e lo Scrum Master non sono inclusi nel conteggio, a meno che non siano anche
loro maiali.\\ 
\linebreak La composizione del Team pu\`o cambiare al termine di uno Sprint. Ogni qual volta un Team
cambia composizione, il livello di produttivit\`a raggiunto tramite l'auto-organizzazione diminuisce. Occorre prestare
particolare attenzione quando si cambia la composizione del Team.