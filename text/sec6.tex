%%%%%%%%%%%%%%%%%%%%%%%%%%%%%%%
% SECTION 6 : Scrum Artifacts %
%%%%%%%%%%%%%%%%%%%%%%%%%%%%%%%
\section*{\color{Blue}{ARTEFATTI}}
\label{sec:artifacts}
Gli Artefatti di Scrum includono il Product Backlog, il  Release Burndown, lo Sprint Backlog e lo Sprint Burndown.

\subsection*{\color{Blue}{PRODUCT BACKLOG E RELEASE BURNDOWN}}
\label{sec:productbacklog}

I requisiti per il prodotto che il Team sta sviluppando sono elencati nel Product Backlog. Il Product Owner \`e responsabile per il Product Backlog, per il suo contenuto, per il fatto che sia a disposizione, e per
la priorit\`a interne allo stesso. Il Product Backlog non \`e  mai completo. Lo sforzo iniziale per svilupparlo serve per definire solo i requisiti noti da subito e quelli compresi meglio.
Il Product Backlog evolve nella misura in cui evolvono il prodotto e l'ambiente in cui esso viene utilizzato. Il Backlog \`e  dinamico nella misura in cui cambia costantemente per 
identificare ci\`o di cui il prodotto ha bisogno per essere appropriato, competitivo, e utile. Fino a quando esiste il prodotto, anche il Product Backlog esiste.

Il Product Backlog rappresenta tutto ci\`o che \`e  necessario per lo sviluppo e il lancio di un prodotto di successo. \`E una lista di tutte le features, funzionalit\`a, tecnologie, miglioramenti, risoluzione di bug
che costituiscono i cambiamenti che verranno fatti al prodotto in una serie di rilasci successivi. Gli elementi del Product Backlog hanno gli attributi Descrizione, Priorit\`a e Stima. La Priorit\`a \`e  calcolata a partire 
da rischio, valore e necessit\`a. Esistono varie tecniche per stimare questi attributi.

\tip{Gli elementi del Product Backlog sono di solito esplicitati attraverso l'uso di User Stories. I Casi d'Uso sono anch'essi appropriati, ma funzionano meglio in contesti di sviluppo di software di particolare criticit\`a, o vitali.}

Il Product Backlog \`e ordinato per priorit\`a. Gli elementi in cima alla lista nel Product Backlog risultano in attivit\`a di sviluppo immediate. Maggiore la priorit\`a, pi\`u urgente la cosa, pi\`u ci si \`e pensato su, e pi\`u consenso esiste 
in relazione al suo valore. La parte del Backlog con priorit\`a pi\`u alta contiene informazioni pi\`u dettagliate rispetto a quella con priorit\`a minore. Le stime fatte sono migliori, perch\`e si appoggiano su una maggior chiarezza e livello di
dettaglio. Minore la priorit\`a, minore il dettaglio, fino a quando non si riesce quasi a distinguere un singolo elemento.

Quando un prodotto viene usato, nella misura in cui il suo valore cresce, e il mercato offre un feedback, il Product Backlog diviene una lista sempre pi\`u lunga ed esaustiva.
I requisiti non smettono mai di cambiare. Il Product Backlog \`e un documento vivo. Dei cambiamenti nei requisiti a livello di business, nelle condizioni di mercato, nella tecnologia, nello staff, danno origine a cambiamenti nel Product Backlog. 
Per minimizzare il lavoro superfluo, solo gli elementi con il livello di priorit\`a pi\`u alta devono essere dettagliati. Gli elementi del Product Backlog che occuperanno il Team nei prossimi Sprint avranno un livello di granularit\`a molto alto,
essendo stati decomposti cos\`i che ciascuno degli elementi pu\`o essere realizzato all'interno della durata di uno Sprint.

\tip{Gli Scrum Team spesso passano il dieci per cento di ogni Sprint per coltivare il Product Backlog e portarlo al livello specificato nella definizione di Product Backlog qui definita. Quando cresce fino a questo livello di granularit\`a, il Product
Backlog possiede in cima alla lista elementi (a priorit\`a massima, valore pi\`u alto) decomposti al punto da poter stare all'interno di un singolo Sprint. Questi sono stati analizzati e ripensati attraverso il processo di coltivazione del Backlog. Quando si tiene lo Sprint Planning Meeting, questi elementi con priorit\`a  massima sono ben compresi e facilmente selezionati.}  

Capita spesso che Team Scrum Multipli lavorino insieme allo stesso prodotto. Un unico Product Backlog \`e usato per descrivere il lavoro da fare sul Prodotto. Viene poi utilizzato un attributo del Product Backlog che raggruppa gli elementi. Il raggruppamento pu\`o avvenire per set di funzionalit\`a, tecnologia, architettura, e viene utilizzato spesso come modalit\`a  di organizzazione del lavoro dagli Scrum Team.

\tip{I Test di Accettazione sono spesso usati come altro attributo del Product Backlog. Possono spesso sostituire descrizioni pi\`u dettagliate di tipo testuale con una descrizione testabile di ci\`o che l'elemento del Product Backlog deve fare quando completato.}

Il grafico Release Burndown tiene traccia della somma dello sforzo rimanente nel tempo, cos\`i come stimato a livello di Product Backlog. Lo sforzo stimato viene espresso in una qualsiasi unit\`a  di misurazione del lavoro concordata dallo Scrum Team e dall'organizzazione in cui il Team opera. Le unit\`a di tempo sono tipicamente gli Sprint.

Le stime per i vari elementi del Product Backlog sono fatte inizialmente durante il Release Planning, e da l\`i in avanti quando gli elementi vengono creati. Durante il processo di coltivazione del Product Backlog le stime sono revisionate e riviste. Tuttavia, possono venire aggiornate in qualsiasi momento. Il Team \`e responsabile per tutte le stime. Il Product Owner pu\`o influenzare il Team aiutandolo a capire i costi-opportunit\`a e a fare delle scelte, ma la stima finale spetta sempre al Team. Il Product Owner mantiene una lista aggiornata del Product Backlog sul 

\tip{}

\tip{}
  
\subsection*{\color{Blue}{SPRINT BACKLOG E SPRINT BURNDOWN}}
\label{sec:sprintbacklog}

\subsection*{\color{Blue}{FATTO}}
\label{sec:done}
