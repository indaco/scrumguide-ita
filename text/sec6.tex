%%%%%%%%%%%%%%%%%%%%%%%%%%%%%%%%%%%%%%%
% SECTION 6: Definizione di ``Fatto'' %
%%%%%%%%%%%%%%%%%%%%%%%%%%%%%%%%%%%%%%%

\section*{\color{Blue}{Definizione di ``Fatto''}}% (fold)
\label{sec:definition_of_done}
\addcontentsline{toc}{section}{Definizione di ``Fatto''}
\pdfbookmark[20]{Definizione di ``Fatto''}{definition_of_done}
Quando l'elemento del Product Backlog o un incremento \`e descritto come ``Fatto'', tutti devono comprendere cosa si intende per 
``Fatto''.  Anche se questo varia in maniera significativo a seconda del Team Scrum, i membri devono avere una comprensione 
condivisa di ci\`o che si intende quando si dice che il lavoro \`e completo, questo aiuta a garantire trasparenza. Questa \`e la 
``Definizione di Fatto'' per il Team ed \`e utilizzata per valutare quando il lavoro \`e completo sull'incremento di prodotto. \newline
\\La stessa definizione guida il Team di sviluppo nel sapere quanti elementi del Product Backlog \`e possibile selezionare 
durante lo Sprint Planning Meeting. Lo scopo di ogni Sprint \`e quello di fornire incrementi di funzionalit\`a potenzialmente 
rilasciabili che aderiscono alla definizione attuale di ``Fatto'' per il Team. \newline
\\I Team di sviluppo forniscono ogni Sprint un incremento di funzionalit\`a del prodotto. Questo incremento \`e utilizzabile  
quindi, un Product Owner pu\`o scegliere di rilasciarlo immediatamente. Ogni incremento \`e additivo a tutti gli incrementi 
precedenti ed \`e testato, garantendo che tutti gli incrementi lavorano insieme. \newline
\\Quando gli Scrum Team maturano si prevede che la loro Definizione di ``Fatto'' si espander\`a per includere criteri pi\`u 
severi per ottenere una qualit\`a superiore.

% section definition_of_done (end)