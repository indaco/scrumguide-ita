%%%%%%%%%%%%%%%%%%%%%%%%%%%%%%%
% SECTION 6 : Scrum Artifacts %
%%%%%%%%%%%%%%%%%%%%%%%%%%%%%%%
\section*{\color{Blue}{ARTEFATTI}}
\label{sec:artifacts}
Gli artefatti includono il Product Backlog, il Release Burndown, lo Sprint Backlog e lo Sprint Burndown.

\subsection*{\color{Blue}{PRODUCT BACKLOG E RELEASE BURNDOWN}}
\label{sec:productbacklog}
I requisiti per il prodotto che il Team (s) \`e in via di sviluppo sono elencati nella Backlog prodotto. Il
proprietario del prodotto \`e responsabile per l'arretrato del prodotto, il suo contenuto, la sua disponibilit\`a, e la
sua priorit\`a. Backlog prodotto non \`e mai completa. Il taglio iniziale di sviluppo che prevede solo la inizialmente
conosciuto e meglio comprendere i requisiti. Il portafoglio ordini del prodotto si evolve come il prodotto e l'ambiente
in cui verr\`a utilizzato evolve. L'arretrato \`e dinamico, nel senso che cambia continuamente di individuare ci\`o che
il prodotto deve essere adeguato, competitivo e utile. Finch\`e un prodotto esiste, esiste anche Backlog prodotto.\\
\linebreak 
Il portafoglio ordini del prodotto rappresenta tutto il necessario per sviluppare e lanciare un prodotto di
successo. Si tratta di un elenco di tutte le caratteristiche, le funzioni, le tecnologie, miglioramenti e correzioni di
bug, che costituiscono le modifiche che saranno apportate al prodotto per le versioni future. Oggetti Backlog Prodotto
hanno gli attributi di una descrizione, la priorit\`a, e stima. La priorit\`a \`e guidato da rischi, il valore, e la
necessit\`a. Ci sono molte tecniche per la valutazione di tali attributi.

\tip{da tradurre}

Backlog prodotto \`e ordinato secondo una priorit\`a. Backlog Top Product priorit\`a unit\`a le attivit\`a di sviluppo
immediato. Maggiore \`e la priorit\`a pi\`u urgente \`e, pi\`u si \`e pensato, e il consenso pi\`u c'\`e per quanto
riguarda il suo valore. Backlog priorit\`a pi\`u alt\`a \`e pi\`u chiaro e ha informazioni pi\`u dettagliate rispetto a
backlog priorit\`a pi\`u bassa. Meglio le stime sono fatte in base al maggior chiarezza e dettaglio maggiore. Pi\`u
bassa \`e la priorit\`a, tanto meno i dettagli, fino a quando non riesce a malapena a distinguere il prodotto.\\
\linebreak

Come un prodotto viene utilizzato, in quanto il suo valore aumenta, e come il mercato fornisce un feedback, arretrato
del prodotto emerge in una lista pi\`u ampia e completa. Requisiti non si fermano mai cambiare. Il Product Backlog \`e
un documento vivo. Cambiamenti nei requisiti di business, le condizioni di mercato, la tecnologia, e il personale
causare modifiche nel backlog prodotto. Per ridurre al minimo le rilavorazioni, solo gli elementi pi\`u alta priorit\`a
devono essere dettagliate fuori. Le voci Backlog prodotto che si occupano delle squadre per il prossimo sprint diversi
sono a grana fine, essendo stato scomposto in modo che ogni elemento pu\`o essere effettuata entro la durata della
Sprint.

\tip{da tradurre}

Multiple squadre Scrum spesso lavorano insieme su uno stesso prodotto. Un backlog di prodotto \`e utilizzato per
descrivere i lavori imminenti sul Prodotto. Un attributo Backlog prodotto che gli elementi dei gruppi viene poi
impiegato. Raggruppamento si pu\`o verificare dal set di funzionalit\`a, la tecnologia, o l'architettura, ed \`e spesso
usato come un modo per organizzare il lavoro con Scrum Team.

\tip{da tradurre}

Il grafico Release Burndown registra la somma di Portafoglio ordini residuo stimato lo sforzo prodotto nel tempo. Lo
sforzo \`e stimato in qualsiasi unit\`a di lavoro del Team Scrum e l'organizzazione sono decise. Le unit\`a di tempo
sono di solito Sprint.\\ 
\linebreak

Stime voce del Product Backlog sono calcolati inizialmente durante la Release Planning, e, successivamente, come
vengono creati. Durante Backlog prodotto grooming sono riesaminato e rivisto. Tuttavia, essi possono essere aggiornati
in qualsiasi momento. Il Team \`e responsabile di tutte le stime. Il Product Owner pu\`o influenzare la squadra,
aiutando a capire e selezionare trade-off, ma la stima finale \`e fatta dal team. Il Product Owner mantiene un elenco
aggiornato dei prodotti Backlog Backlog Release Burndown Posted in ogni momento. La linea di tendenza si possono trarre
sulla base del cambiamento di lavoro rimanente.

\tip{da tradurre}

\subsection*{\color{Blue}{SPRINT BACKLOG E SPRINT BURNDOWN}}
\label{sec:sprintbacklog}
Lo Sprint Backlog comprende i compiti del team esegue a sua volta elementi Backlog prodotto in un ''fatto'' di
incremento. Molti sono sviluppate durante la Sprint Planning Meeting. \`E tutto il lavoro che il team identifica come
necessario per soddisfare l'obiettivo Sprint. Oggetti Backlog Sprint deve essere scomposto. La decomposizione \`e
sufficiente modo che le modifiche in corso pu\`o essere compreso nel Daily Scrum.

\tip{da tradurre}

Il team di modifica Backlog Sprint tutta la Sprint, cos\`i come Backlog Sprint emergenti durante la Sprint. Come si
arriva in singole attivit\`a, si pu\`o scoprire che i compiti pi\`u o meno sono necessari, o che un determinato compito
sar\`a pi\`u o meno tempo di quanto fosse stato previsto. Come nuovo lavoro \`e necessario, il Team si aggiunge al
residuo Sprint. Come compiti sono manipolate o completati, le ore di lavoro stimato rimanente per ogni compito \`e
aggiornato. Quando le attivit\`a sono ritenute inutili, che vengono rimossi. Solo il team pu\`o cambiare il suo corso
di uno Sprint Backlog. Solo il team possono modificare i contenuti o le stime. Lo Sprint Backlog ha una grande
visibilit\`a, immagine in tempo reale del lavoro che il team ha in programma di realizzare nel corso della Sprint, ed
appartiene esclusivamente al team.\\ 
\linebreak

Backlog Sprint Burndown \`e un grafico della quantit\`a di lavoro Portafoglio ordini residuo in un Sprint Sprint
attraverso il tempo nella Sprint. Per creare questo grafico, determinare quanto lavoro resta sommando il ritardo stime
ogni giorno della Sprint. La quantit\`a di lavoro rimanente per un Sprint \`e la somma del lavoro rimanente per tutti
Backlog Sprint. Tenere traccia di queste somme di giorno e li usa per creare un grafico che mostra i restanti lavori
nel corso del tempo. Tracciando una linea attraverso i punti del grafico, il team in grado di gestire i suoi progressi
nel completamento di un lavoro di Sprint. Durata non \`e considerato in Scrum. Restanti lavori \`e la data sono le sole
variabili di interesse.\\ 
\linebreak 

Una delle regole Scrum appartiene alla fine di ogni Sprint, che \`e quello di fornire incrementi di funzionalit\`a
potenzialmente rilasciabile che aderisca ad una definizione operativa di ''fatto''.

\tip{da tradurre}

\subsection*{\color{Blue}{FATTO}}
\label{sec:done}
Scrum richiede squadre di costruire un incremento delle funzionalit\`a del prodotto ogni Sprint. Tale incremento deve
essere potenzialmente rilasciabile, per il prodotto proprietario pu\`o scegliere di applicare subito la funzionalit\`a.
Per fare ci\`o, l'incremento deve essere una fetta completa del prodotto. Essa deve essere ''fatto.'' Ogni incremento
dovrebbe essere a tutti gli additivi incrementi precedenti e testato, assicurando che tutti incrementi lavorare
insieme.

\tip{da tradurre}

Nello sviluppo del prodotto, affermando che la funzionalit\`a \`e fatto potrebbe portare qualcuno a ritenere che, in
almeno pulito codice, il refactoring, unit test, costruito e testato l'accettazione. Qualcun altro potrebbe supporre
solo che il codice \`e stato costruito. Se tutti non sa che la definizione di ''fatto'' \`e, gli altri due le gambe di
controllo del processo empirico non funzionano. Quando qualcuno descrive come fare qualcosa, ognuno deve capire cosa si
intende fare.\\
\linebreak

Fatto definisce ci\`o che il team intende quando si impegna a ''fare'' un elemento di backlog prodotto in uno sprint.
Alcuni prodotti non contengono la documentazione, quindi la definizione di ''fatto'' non comprende la documentazione. A
completamente ''fatto'' incremento include tutte le analisi, progettazione, refactoring, programmazione, documentazione
e analisi per l'incremento e tutti gli elementi backlog prodotto in incremento. Test comprende unit\`a, di sistema,
l'utente, e di regressione, così come non le prove funzionali, come le prestazioni, stabilit\`a, sicurezza e
integrazione. Fatto include qualsiasi internazionalizzazione. Alcune squadre non sono ancora in grado di comprendere
tutto il necessario per l'esecuzione nella loro definizione di fatto. Questo deve essere chiaro per il proprietario del
prodotto. Questo lavoro rimanente dovr\`a essere fatta prima che il prodotto pu\`o essere implementato e utilizzato.