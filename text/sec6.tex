%%%%%%%%%%%%%%%%%%%%%%%%%%%%%%%
% SECTION 6 : Scrum Artifacts %
%%%%%%%%%%%%%%%%%%%%%%%%%%%%%%%
\section*{\color{Blue}{ARTEFATTI}}
\label{sec:artifacts}
Gli Artefatti di Scrum includono il Product Backlog, il  Release Burndown, lo Sprint Backlog e lo Sprint Burndown.

\subsection*{\color{Blue}{PRODUCT BACKLOG E RELEASE BURNDOWN}}
\label{sec:productbacklog}

I requisiti per il prodotto che il Team sta sviluppando sono elencati nel Product Backlog. Il Product Owner \`e responsabile per il Product Backlog, per il suo contenuto, per il fatto che sia a disposizione, e per
la priorit\`a interne allo stesso. Il Product Backlog non \`e  mai completo. Lo sforzo iniziale per svilupparlo serve per definire solo i requisiti noti da subito e quelli compresi meglio.
Il Product Backlog evolve nella misura in cui evolvono il prodotto e l'ambiente in cui esso viene utilizzato. Il Backlog \`e  dinamico nella misura in cui cambia costantemente per 
identificare ci\`o di cui il prodotto ha bisogno per essere appropriato, competitivo, e utile. Fino a quando esiste il prodotto, anche il Product Backlog esiste.

Il Product Backlog rappresenta tutto ci\`o che \`e  necessario per lo sviluppo e il lancio di un prodotto di successo. \`E una lista di tutte le features, funzionalit\`a, tecnologie, miglioramenti, risoluzione di bug
che costituiscono i cambiamenti che verranno fatti al prodotto in una serie di rilasci successivi. Gli elementi del Product Backlog hanno gli attributi Descrizione, Priorit\`a e Stima. La Priorit\`a \`e  calcolata a partire 
da rischio, valore e necessit\`a. Esistono varie tecniche per stimare questi attributi.

\tip{Gli elementi del Product Backlog sono di solito esplicitati attraverso l'uso di User Stories. I Casi d'Uso sono anch'essi appropriati, ma funzionano meglio in contesti di sviluppo di software di particolare criticit\`a, o vitali.}

Il Product Backlog \`e ordinato per priorit\`a. Gli elementi in cima alla lista nel Product Backlog risultano in attivit\`a di sviluppo immediate. Maggiore la priorit\`a, pi\`u urgente la cosa, pi\`u ci si \`e pensato su, e pi\`u consenso esiste 
in relazione al suo valore. La parte del Backlog con priorit\`a pi\`u alta contiene informazioni pi\`u dettagliate rispetto a quella con priorit\`a minore. Le stime fatte sono migliori, perch\`e si appoggiano su una maggior chiarezza e livello di
dettaglio. Minore la priorit\`a, minore il dettaglio, fino a quando non si riesce quasi a distinguere un singolo elemento.

Quando un prodotto viene usato, nella misura in cui il suo valore cresce, e il mercato offre un feedback, il Product Backlog diviene una lista sempre pi\`u lunga ed esaustiva.
I requisiti non smettono mai di cambiare. Il Product Backlog \`e un documento vivo. Dei cambiamenti nei requisiti a livello di business, nelle condizioni di mercato, nella tecnologia, nello staff, danno origine a cambiamenti nel Product Backlog. 
Per minimizzare il lavoro superfluo, solo gli elementi con il livello di priorit\`a pi\`u alta devono essere dettagliati. Gli elementi del Product Backlog che occuperanno il Team nei prossimi Sprint avranno un livello di granularit\`a molto alto,
essendo stati decomposti cos\`i che ciascuno degli elementi pu\`o essere realizzato all'interno della durata di uno Sprint.

\tip{Gli Scrum Team spesso passano il dieci per cento di ogni Sprint per coltivare il Product Backlog e portarlo al livello specificato nella definizione di Product Backlog qui definita. Quando cresce fino a questo livello di granularit\`a, il Product
Backlog possiede in cima alla lista elementi (a priorit\`a massima, valore pi\`u alto) decomposti al punto da poter stare all'interno di un singolo Sprint. Questi sono stati analizzati e ripensati attraverso il processo di coltivazione del Backlog. Quando si tiene lo Sprint Planning Meeting, questi elementi con priorit\`a  massima sono ben compresi e facilmente selezionati.}  

Capita spesso che Team Scrum Multipli lavorino insieme allo stesso prodotto. Un unico Product Backlog \`e usato per descrivere il lavoro da fare sul Prodotto. Viene poi utilizzato un attributo del Product Backlog che raggruppa gli elementi. Il raggruppamento pu\`o avvenire per set di funzionalit\`a, tecnologia, architettura, e viene utilizzato spesso come modalit\`a  di organizzazione del lavoro dagli Scrum Team.

\tip{I Test di Accettazione sono spesso usati come altro attributo del Product Backlog. Possono spesso sostituire descrizioni pi\`u dettagliate di tipo testuale con una descrizione testabile di ci\`o che l'elemento del Product Backlog deve fare quando completato.}

Il grafico Release Burndown tiene traccia della somma dello sforzo rimanente nel tempo, cos\`i come stimato a livello di Product Backlog. Lo sforzo stimato viene espresso in una qualsiasi unit\`a  di misurazione del lavoro concordata dallo Scrum Team e dall'organizzazione in cui il Team opera. Le unit\`a di tempo sono tipicamente gli Sprint.

Le stime per i vari elementi del Product Backlog sono fatte inizialmente durante il Release Planning, e da l\`i in avanti quando gli elementi vengono creati. Durante il processo di coltivazione del Product Backlog le stime sono revisionate e riviste. Tuttavia, possono venire aggiornate in qualsiasi momento. Il Team \`e responsabile per tutte le stime. Il Product Owner pu\`o influenzare il Team aiutandolo a capire i costi-opportunit\`a e a fare delle scelte, ma la stima finale spetta sempre al Team. Il Product Owner mantiene una lista aggiornata del Product Backlog sul Release Backlog Burndown, che \`e sempre disponibile. \`E possibile tracciare una linea di tendenza basandosi sui cambiamenti nel lavoro che manca.  

\tip{In alcune organizzazioni, viene aggiunto pi\`u lavoro al backlog di quanto sia effettivamente poi completato. Questo pu\`o creare una linea di tendenza piatta, o addirittura che punta verso l'alto. Per compensare ci\`o, e per mantenere trasparenza, \`e possibile aggiungere un nuovo pavimento quando del lavoro viene aggiunto o sottratto. Il pavimento dovrebbe aggiungere o rimuovere solo cambiamenti signicativi, e deve essere ben documentato.}

\tip{La linea di tendenza pu\`o essere inaffidabile per i primi due o tre Sprint di una release, a meno che il Team abbia lavorato assieme gi\`a in precedenza, conosca bene il prodotto, e comprenda bene la tecnologia sottostante.}
  
\subsection*{\color{Blue}{SPRINT BACKLOG E SPRINT BURNDOWN}}
\label{sec:sprintbacklog}

Lo Sprint Backlog \`e composto dall'insieme di attivit\`a che il Team fa, per trasformare gli elementi del Product Backlog in un incremento ''fatto''. Molti vengono sviluppati durante lo Sprint Plannning Meeting. \`E in pratica tutto il lavoro che il Team identifica come necessario per raggiungere l'obiettivo dello Sprint. Gli elementi dello Sprint Backlog devono essere scomposti. Il livello di scomposizione \`e sufficiente nel momento in cui i cambiamenti in corso possono essere capiti durante il Daily Scrum.

Il Team modifica lo Sprint Backlog durante tutto lo Sprint, cos\`i come ci sono Sprint Backlog che emergono durante lo Sprint. Quando si tratta di attivit\`a indidivuali, \`e possibile scoprire che le attivit\`a necessarie sono ancora di pi\`u, o invece sono meno, o che una certa attivit\`a prender\`a pi\`u tempo, o meno, di quanto previsto. Nella misura in cui \`e richiesto pi\`u lavoro, il Team lo aggiunge allo Sprint Backlog. Nella misura in cui si lavora a delle attivit\`a, o le stesse sono completate, la stima di ore restanti, necessarie per completare quelle attivit\`a, viene aggiornata.  Quando delle attivit\`a vengono considerate non pi\`u necessarie, vengono rimosse. Solamente il Team pu\`o cambiare il suo Sprint Backlog durante uno Sprint. Solo il Team pu\`o cambiare il contenuto delle stime. Lo Sprint Backlog \`e un'immagine altamente visibile, in tempo reale, del lavoro che il Team prevede di completare durante lo Sprint, e appartiene esclusivamente al Team.

Lo Sprint Backlog Burndown \`e un grafico dell'ammontare di lavoro rimanente sullo Sprint Backlog, in uno Sprint, durante tutto il tempo dello Sprint. Per creare questo grafico, determinate quanto lavoro vi rimane da fare sommando le stime del Backlog ogni giorno dello Sprint. L'ammontare di lavoro restante per uno Sprint \`e la somma del lavoro che manca per l'intero Sprint Backlog. Tenete traccia di queste somme giorno dopo giorno, e usatele per creare un grafico che mostri il lavoro restante nel tempo. Tracciando una linea che passa per i punti sul grafico, il Team pu\`o visualizzare i propri progressi nel completamento del lavoro di uno Sprint. La durata non viene considerata, in Scrum. Il lavoro rimanente e la data sono le uniche variabili di interesse. 

\tip{Ogni qual volta possibile, disegnate a mano il grafico Burndown su un grosso foglio di carta, mostrato nella zona di lavoro del Team. \`E più probabile che il Team presti attenzione a un grafico disegnato, grande e ben visibile, che a un grafico fatto in Excel o con un altro strumento software. }

Una delle regole di Scrum riguarda l'obiettivo di ogni Sprint, che \`e di realizzare incrementi potenzialmente consegnabili di funzionalit\`a, conformi a una definizione operativa di  ''fatto''.

\subsection*{\color{Blue}{FATTO}}
\label{sec:done}

\tip{Il lavoro ''non fatto'' viene spesso accumulato in un elemento del Product Backlog chiamato ''lavoro non fatto''. Nella misura in cui questo lavoro si accumula, il Product Backlog Burndown resta pi\`u accurato che se non lo fosse.}

Scrum richiede che i Team costruiscano un incremento di funzionalit\`a nel prodotto per ogni Sprint. Questi incrementi devono essere potenzialmente consegnabili, perch\`e  il Product Owner pu\`o scegliere di implementare immediatamente la funzionalit\`a. Per fare questo, l'incremento deve essere una porzione completa di prodotto. Deve essere  ''fatto''. Ogni incremento dovrebbe essere addizionale in relazione a tutti gli incrementi precedenti e testato estensivamente, assicurandosi che tutti gli incrementi funzionino bene assieme.

Nello sviluppo di un prodotto, asserire che una funzionalit\`a è pronta, cio\`e fatta, pu\`o portare qualcuno ad assumere che sia quanto meno programmata in modo chiaro, rifattorizzata, abbia passato le fasi di unit test, compilazione e acceptance test. Qualcun altro pu\`o assumere che si parli solo del fatto che il codice \`e stato compilato. Se qualcuno non sa quale \`e la definizione di ''fatto'', le altre due colonne del processo di controllo empirico non funzionano. Quando qualcuno descrive qualcosa come fatto, tutti devono capire che cosa significa fatto.

Fatto definisce cosa intende il Team quando si impegna a ''fare'' un elemento del Product Backlog in uno Sprint. Alcuni prodotti non contengono documentazione, quindi la definizione di ''fatto'' non comprende la documentazione. Un incremento completamente ''fatto'' include tutta la parte di analisi, design, refactoring, programmazione, documentazione e test per quell'incremento e per tutti gli elementi del Product Backlog compresi in quell'incremento. I test includono unit test, test di sistema, livello utente e di regressione, cos\`i come test non funzionali quali test di performance, stabilit\`a, sicurezza, e integrazione. Fatto pu\`o comprendere anche l'internazionalizzazione. Alcuni Team tuttavia non riescono a includere tutto ci\`o che \`e richiesto per l'implementazione nella loro definizione di fatto. Questa cosa deve essere chiara al Product Owner. Questo lavoro rimanente dovr\`a essere fatto prima che il prodotto possa essere implementato e usato.
