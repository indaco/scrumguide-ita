%%%%%%%%%%%%%%%%%%%%%%%%%%%%%%%%%%%%%%%%%
% SECTION 2 : La Teoria di Scrum, Scrum	%
%%%%%%%%%%%%%%%%%%%%%%%%%%%%%%%%%%%%%%%%%

\section*{\color{Blue}{La Teoria di Scrum}}
\label{sec:scrum_theory}
\addcontentsline{toc}{section}{La Teoria di Scrum}
Scrum si basa sulla teoria dei controlli empirici di processo o empirismo. L'empirismo afferma che la conoscenza deriva dall'esperienza e 
che le decisioni si basano su ci\`o che \`e noto. Scrum utilizza un metodo iterativo ed un approccio incrementale per ottimizzare la 
prevedibilit\`a ed il controllo del rischio. 
\newline
\\Sono tre i pilastri che sostengono ogni implementazione del controllo empirico di processo.

\subsection*{\color{SteelBlue}{Trasparenza}}
\label{sec:transparency}
Gli aspetti significativi  del processo devono essere visibili ai responsabili del lavoro. La trasparenza richiede che quegli aspetti siano 
definiti da uno standard comune in modo tale che gli osservatori condividano una comune comprensione di ci\`o che viene visto.
\newline
\\Per esempio:
\begin{itemize}
	\item Un linguaggio comune di riferimento al processo deve essere condiviso da tutti i partecipanti; e,
	\item Una definizione comune di ``Fatto'' \footnote[1]{Vedere ``Definizione di Fatto'', pag. \pageref{sec:definition_of_done}} deve essere condivisa da chi esegue il lavoro e da chi deve accettare il lavoro.
\end{itemize}

\subsection*{\color{SteelBlue}{Ispezione}}
\label{sec:inspection}
Gli utenti di Scrum devono ispezionare frequentamente gli artefetti di Scrum e il progresso verso un obiettivo per rilevare le variazioni
indesiderate. Le loro ispezioni non dovrebbero essere cos\`i frequenti da superare le soglie di tolleranza del processo all'ispezioni
rappresentare un'interruzione del lavoro. Le ispezioni sono pi\`u utili quando diligentemente eseguite da ispettori qualificati in 
corrispondenza del punto di lavoro.

\subsection*{\color{SteelBlue}{Adattamento}}
Se chi  ispeziona verifica che uno o pi\`u aspetti del processo sono al di fuori dei limiti accettabili  e che il prodotto finale non
potr\`a essere accettato, deve regolare il processo o il materiale lavorato.  La regolazione  deve essere  effettuata il  pi\`u rapidamente
possibile per ridurre al minimo l'ulteriore scarto.

Scrum prescrive quattro occasioni formali per l'ispezione e l'adattamento, come descritto nella sezione ``Gli Eventi di Scrum'' di questo 
documento.
\begin{itemize}
\item Scrum Planning Meeting
\item Daily Scrum
\item Sprint Review Meeting
\item Sprint Retrospective
\end{itemize}


\section*{\color{Blue}{Scrum}}
\label{sec:scrum}
\addcontentsline{toc}{section}{Scrum}
Scrum \`e un framework strutturato per supportare lo sviluppo di un prodotto complesso. Il framework Scrum \`e costituito dai Scrum Team e 
dai ruoli, eventi, artifatti e regole ad essi associati. Ogni componente del framewrok serve ad uno specifico scopo ed \`e essenziale per il 
successo e l'utilizzo di Scrum.
\newline
\\ Le regole di Scrum legano insieme gli eventi, i ruoli e gli artefatti governando le relazioni e le interazioni tra essi e sono descritte 
in tutto il corpo di questo documento.