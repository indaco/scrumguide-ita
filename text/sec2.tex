%%%%%%%%%%%%%%%%%%%%%%%%%%%%%%%%%%%%%%%%%%%%%%%%%
% SECTION 2 : Purpose, Scrum Theory and pillars %
%%%%%%%%%%%%%%%%%%%%%%%%%%%%%%%%%%%%%%%%%%%%%%%%%
\newpage
\section*{\color{Blue}{LO SCOPO}}
\label{sec:purpose}
Scrum  \`e  stato  impiegato  per  sviluppare   prodotti
complessi sin dai  primi anni '90.  Questo documento descrive  come usare Scrum
per costruire prodotti.
Scrum non \`e un processo o una tecnica bens\`i un  framework all'interno del quale  possiamo utilizzare
vari processi e varie tecniche. Il ruolo di Scrum \`e quello di far emergere l'efficacia
delle  pratiche di  sviluppo adottate,  in modo  da migliorarle  e fornendo  un
framework con il quale sviluppare soluzioni complesse.

\section*{\color{Blue}{LA TEORIA  DI SCRUM}}
\label{sec:scrum_theory}
Scrum si basa sulla teoria dei controlli empirici di processo. Utilizza un
metodo iterativo ed un  approccio incrementale per ottimizzare la prevedibilit\`a
ed  il  controllo  del  rischio.  Sono  tre  i  pilastri  che  sostengono   ogni
implementazione  del  controllo empirico  di  processo.

\subsection*{\color{Blue}{LA PRIMA  TAPPA  \`E LA TRASPARENZA}}
\label{sec:transparency}
La  trasparenza  garantisce  che   gli  aspetti  del  processo   che
influenzano il risultato siano visibili a coloro che gestiscono i risultati. Non
solo questi aspetti  devono essere trasparenti,  ma anche ci\`o  che si vede  deve
essere noto.  Cio\`e, quando  qualcuno controlla  lo stato  del processo e crede che
qualcosa sia fatto, la nostra definizione di fatto deve equivalere alla sua.

\subsection*{\color{Blue}{LA  SECONDA  TAPPA \`E  L'ISPEZIONE}}
\label{sec:inspection}
I  vari aspetti  del  processo devono essere controllati abbastanza spesso  in modo che  variazioni 
inaccettabili nello stesso possano essere individuate. La frequenza delle ispezioni deve  prendere
in considerazione  il fatto  che tutti  i processi cambiano nel momento stesso in cui si esegue un'
ispezione. Un'enigma si verifica  quando la frequenza di richieste  di ispezione
supera  la  tolleranza al  controllo  del processo.  Fortunatamente,  questo non
sembra  esser vero per lo sviluppo di software. L'altro  fattore \`e l'abilit\`a e la
diligenza di chi ispeziona i risultati del lavoro.

\subsection*{\color{Blue}{LA TERZA  TAPPA \`E  L'ADATTAMENTO}}
Se chi  ispeziona verifica che uno o pi\`u aspetti del processo sono al di fuori dei limiti accettabili 
e che il prodotto finale non potr\`a essere accettato, deve regolare il processo o il
materiale lavorato.  La regolazione  deve essere  effettuata il  pi\`u rapidamente
possibile per ridurre al minimo l'ulteriore scarto.

In Scrum ci sono pi\`u momenti dedicati all'ispezione e all'adattamento:
\begin{description}
\item[- l'incontro Daily  Scrum:]
	usato per controllare i progressi verso l'obiettivo dello Sprint e per procedere ad
	adattamenti  che  ottimizzano il  valore  del giorno  successivo  di lavoro.
\item[- gli incontri Sprint Review e  Planning Review:]
	usati per ispezionare i progressi  verso l'obiettivo di rilascio e per procedere con gli
	adattamenti che ottimizzano il valore del prossimo Sprint.
\item[- l'incontro Sprint Retrospective:]
	usato  per  esaminare il passato Sprint e determinare quali
	adattamenti potranno renderne il prossimo pi\`u produttivo, appagante  e
	divertente.
\end{description}