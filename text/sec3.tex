%%%%%%%%%%%%%%%%%%%%%%%%%%
% SECTION 3 : Scrum Team %
%%%%%%%%%%%%%%%%%%%%%%%%%%

\section*{\color{Blue}{Scrum Team}}% (fold)
\label{sec:team}
\addcontentsline{toc}{section}{Scrum Team}
\pdfbookmark[6]{Scrum Team}{team}
Il Team Scrum \`e formato dal Product Owner, Il Team e uno Scrum Master. I Team  Scrum sono auto-organizzati e 
cross-funzionali. I Team auto-organizzati scelgono come meglio compiere il lavoro invece di essere diretti da altri al di fuori 
del team. I Team cross-funzionali hanno tutte le competenze necessarie per realizzare il lavoro senza dover dipendere da 
nessuno al di fuori del team. Il modello di team in Scrum \`e progettato per ottimizzare la flessibilità, la creativit\`a e la 
produttivit\`a. \newline

\\I Team Scrum rilasciano i prodotti in modo iterativo ed incrementale, massimizzando le opportunit\`a di feedback. I rilasci 
incrementali di prodotto ``Fatto'' garantiscono che una versione potenzialmente utile del prodotto funzionante sia sempre 
disponibile.

\subsection*{\color{SteelBlue}{Il Product Owner}}% (fold)
\label{sec:productowner}
\addcontentsline{toc}{subsection}{Il Product Owner}
\pdfbookmark[7]{Il Product Owner}{productowner}
Il Product Owner ha la responsabilit\`a  di massimizzare il valore del prodotto e del lavoro svolto dal Team. Come questo \`e 
fatto pu\`o variare di molto a seconda dell'organizzazione, dei Team Scrum e degli individui. \newline 

\\Il Product Owner ha la responsabilit\`a esclusiva di gestione del Product Backlog. Tale gestione comporta che:

\begin{itemize}
	\item Gli elementi del Product Backlog sono espressi in modo chiaro;
	\item Gli elementi del Product Backlog sono ordinati per raggiungere meglio gli obiettivi e missioni;
	\item Il valore del lavoro svolto dal Team di Sviluppo \`e garantito;
	\item Il Product Backlog \`e visibile, trasparente e chiaro a tutti e mostri cosa il Team Scrum lavorerà 
	successivamente; e,
	\item Gli elementi del Product Backlog sono compresi al necessario livello dal Team di Sviluppo.
\end{itemize}

\noindent Il lavoro sopra elencato pu\`o esser fatto dal Product Owner o dal Team di Sviluppo. Tuttavia, il Product Owner 
rimane il responsabile.  \newline

\\Il Product Owner \`e una persona, non un comitato. Il Product Owner pu\`o rappresentare un desiderio di un comitato nel 
Product Backlog, ma chiunque voglia cambiare la priorit\`a di un elemento deve convincere il Product Owner. \newline

\\Affinch\`e il Product Owner abbia successo, all'interno dell'organizzazione tutti devono rispettare le sue decisioni. Le 
decisioni del Product Owner sono visibili nel contenuto e nell'ordine delle priorit\`a del Product Backlog. A nessuno \`e 
permesso dire al Team di lavorare con un diverso ordine di priorit\`a, e i Team non sono autorizzati ad ascoltare chi sostiene 
il contrario. 
% subsection productowner (end)

\subsection*{\color{SteelBlue}{Il Team di Sviluppo}}% (fold)
\label{sec:development_team}
\addcontentsline{toc}{subsection}{Il Team di Sviluppo}
\pdfbookmark[8]{Development Team}{development_team}
Il Team di Sviluppo \`e costituito da professionisti che lavorano per produrre un incremento potenzialmente rilasciabile di 
prodotto ``Fatto'' alla fine di ogni Sprint. Soltanto i membri del Team di Sviluppo creano l'incremento. \newline

\\I Team di Sviluppo sono strutturati e autorizzati dalle organizzazioni per organizzare e gestire il proprio lavoro. La 
sinergia risultante ottimizza l'efficienza del Team di Sviluppo e l'efficacia. I Team di Sviluppo hanno le seguenti 
caratteristiche:

\begin{itemize}
	\item Sono auto-organizzati. Nessuno (neanche lo Scrum Master) dice al Team di Sviluppo come trasformare il Product Backlog 
	in Incrementi di funzionalit\`a potenzialmente rilasciabili;
	\item I Team di Sviluppo sono cross-funzionali, con tutte le competenze come team necessarie a creare un incremento di 
	prodotto;
	\item Scrum non riconosce alcun titolo ai membri del Team di Sviluppo al di fuori di Sviluppatore, indipendentemente dal 
	lavoro eseguito dalla persona; non ci sono eccezioni a questa regola;
	\item I singoli membri dei Team di Sviluppo hanno competenze specialistiche e aree di focus, ma la responsabilit\`a \`e del 
	Team di Sviluppo nel suo complesso; e,
	\item I Team di Sviluppo non contengono sotto-team dedicati a particolari domini come il testing o la business analysys.
\end{itemize}

\subsubsection*{\color{SteelBlue}{Dimensioni del Team di Sviluppo}} % (fold)
\label{ssub:development_team_size}
La dimensione ottimale del Team di Sviluppo \`e piccola abbastanza da rimanere agile e grande abbastanza per conpletare il 
lavoro  significativo. Avere Team di Sviluppo con meno di tre membri dimunuisce l'interazione e comporta un minore guadagno in 
termini di produttivit\`a. Team di Sviluppo pi\`u piccoli possono incontrare vincoli di competenza durante lo Sprint, rendendo  
impossibile al Team di Sviluppo di fornire un incremento potenzialmente rilasciabile. Avere pi\`u di nove membri rende il 
coordinamento molto oneroso. Team di Svilupo di grandi dimensioni generano troppa complessit\`a che non \`e possibile gestire 
con un processo empirico. Il Product Owner e lo Scrum Master non sono inclusi nel conteggio a meno che non stiano anche loro 
eseguendo il lavoro del Product Backlog.
% subsubsection development_team_size (end)
%% subsection development_team (end)

\subsection*{\color{SteelBlue}{Lo Scrum Master}}% (fold)
\label{sec:scrummaster}
\addcontentsline{toc}{subsection}{Lo Scrum Master}
\pdfbookmark[9]{Lo Scrum Master}{scrummaster}
Lo Scrum Master \`e responsabile di assicurare che Scrum \`e compreso e approvato. Gli Scrum Master fanno questo assicurandosi 
che il Team Scrum aderisca ai valori, alle pratiche e alle regole di Scrum. \`E un servo-leader per il Team Scrum. \newline 
Lo Scrum Master aiuta coloro al di fuori del Team Scrum a capire quali delle loro interazioni con il Team Scrum sono utili e  
quali no. Aiuta tutti a modificare queste interazioni per massimizzare il valore creato dal Team Scrum.

\subsubsection*{\color{SteelBlue}{Servizio dello Scrum Master per il Product Owner}} % (fold)
\label{ssub:sm_service_to_po}
Lo Scrum Master lavora per il Product Owner in diversi modi, tra cui:
\begin{itemize}
	\item Trovare le tecniche per una gestione efficace del Product Management;
	\item Comunicare con chiarezza la visione, gli obiettivi e gli elementi del Product Backlog al Team di Sviluppo;
	\item Insegnare al Team di Sviluppo come creare gli elementi del Product Backlog in modo chiaro e conciso
	\item Comprendere la pianificazione a lungo termine del prodotto in un ambiente empirico;
	\item Capire e praticare l'agilit\`a; e,
	\item Facilitare gli eventi Scrum come richiesto o necessario.
\end{itemize}
% subsubsection sm_service_to_po (end)

\subsubsection*{\color{SteelBlue}{Servizio dello Scrum Master per il Team Scrum}} % (fold)
\label{ssub:sm_service_to_team}
Lo Scrum Master lavora per il Team Scrum in diversi modi, tra cui:
\begin{itemize}
	\item Coaching al Team di Sviluppo per l'auto-organizzazione e la cross-funzionalit\`a;
	\item Insegnare e guidare il Team di Sviluppo a creare prodotti di alto valore;
	\item Eliminare gli ostacoli al progresso del Team di Sviluppo
	\item Facilitare gli eventi Scrum come richiesto o necessario; e,
	\item Coaching al Team di Sviluppo in ambienti organizzativi in cui Scrum non \`e ancora adottato e compreso.
\end{itemize}
% subsubsection sm_service_to_team (end)

\subsubsection*{\color{SteelBlue}{Servizio dello Scrum Master per l'Organizzazione}} % (fold)
\label{ssub:sm_service_to_org}
Lo Scrum Master lavora per l'Organizzazione in diversi modi, tra cui:
\begin{itemize}
	\item Coaching all'organizzazione nel processo di adozione di Scrum;
	\item Pianificare l'implementazioni di Scrum all'interno dell'organizzazione;
	\item Aiutare i dipendenti e gli stakeholders a capire e mettere in atto Scrum e lo sviluppo empirico di prodotto;
	\item Provocare il cambiamento che aumenta la produttivit\`a del Team Scrum; e,
	\item Lavorare con altri Scrum Master per aumentare l'efficacia dell'applicazione di Scrum nell'organizzazione.
\end{itemize}
% subsubsection sm_service_to_org (end)
%% subsection scrummaster (end)

%%% subsection team (end)