%%%%%%%%%%%%%%%%%%%%%%%%%%%%%
% SECTION 3 : Scrum Content %
%%%%%%%%%%%%%%%%%%%%%%%%%%%%%
\section*{\color{Blue}{LA SOSTANZA DI SCRUM}}
Il framework Scrum  \`e costituito da una serie  di
Scrum Team e dai ruoli a essi associati; Time-Box, Artefatti e Regole.
I  Scrum Team sono concepiti per   ottimizzare la  flessibilit\`a e  la produttivit\`a. Proprio per questo
sono auto-organizzati, cross-funzionali e lavorano basandosi su iterazioni.
Ogni Scrum Team ha tre ruoli: 

\begin{enumerate}
\item Scrum Master: responsabile del fatto che il processo venga compreso e seguito;
\item Product Owner: responsabile di massimizzare il valore
del lavoro  che il  Team Scrum  fa; 
\item Team: fa  il lavoro. Il Team consiste di
sviluppatori con  tutte le  competenze necessarie  per tradurre le richieste  del
Product Owner in  una
porzione potenzialmente rilasciabile di prodotto entro  la fine  dello Sprint.

\end{enumerate}

Scrum impiega intervalli di tempo definiti (Time-Boxes) per creare regolarit\`a. Gli elementi di Scrum che sottostanno a
questa logica comprendono:
\begin{itemize}
\item Release  Planning  Meeting 
\item Sprint Planning  Meeting
\item Sprint
\item Daily Scrum Meeting
\item Sprint Review
\item Sprint Retrospective
\end{itemize}

Il cuore di Scrum \`e lo \textbf{Sprint}, cio\`e una iterazione di un mese o meno, di lunghezza coerente per
tutta la durata del progetto.

Tutti gli sprint utilizzano  lo stesso framework Scrum  e tutti
gli sprint  consegnano un  incremento del  prodotto finale  che \`e potenzialmente
rilasciabile.  Uno  Sprint inizia  immediatamente  dopo la fine del precedente.

Scrum  utilizza  quattro artefatti principali:\begin{itemize}
\item[-] Product Backlog: \`e l'elenco di tutto ci\`o che potrebbe essere necessario al prodotto, ordinato per
priorit\`a;
\item[-] Sprint Backlog: \`e la lista di compiti necessari a trasformare la parte di Product Backlog relativa a uno
Sprint in un incremento di prodotto potenzialmente rilasciabile;
\item[-] Burndown:  misura la quantit\`a residua di un Backlog nel corso del tempo;
\item[-] Release Burndown: misura la quantit\`a residua di Product Backlog rispetto al piano di rilascio (Release Plan);
\item[-] Sprint Burndown: misura la quantit\`a  di elementi residui di uno Sprint Backlog
nel corso  di uno  Sprint.
\end{itemize}

Vi sono delle regole che legano insieme i time-boxes, i ruoli e gli artefatti dello Scrum. Queste regole vengono
descritte nel presente documento. Per esempio, \`e regola dello Scrum che solo i membri del Team - le persone impegnate
a trasformare il Product Backlog in un incremento di prodotto - possono parlare durante un Daily Scrum Meeting. Le
modalit\`a di implementazione di Scrum che non sono regole, ma consigli, saranno presentati in riquadri specifici,
''suggerimenti''.
\vspace{0.4cm}

\tip{Quando le regole non sono esplicitate, chi usa Scrum si aspetta di capire cosa fare. Non cercare di
trovare una soluzione perfetta, perch\`e le cose di solito cambiano rapidamente. Al contrario, prova qualcosa e vedi se
funziona. I meccanismi di natura empirica di Scrum, di ispezione-adeguamento, ti guideranno.}