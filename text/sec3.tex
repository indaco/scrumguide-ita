%%%%%%%%%%%%%%%%%%%%%%%%%%%%%
% SECTION 3 : Scrum Content %
%%%%%%%%%%%%%%%%%%%%%%%%%%%%%
\section*{\color{Blue}{LA SOSTANZA DI SCRUM}}
Il framework Scrum  \`e costituito da una serie  di
Scrum Team e dai ruoli ad essi associati: Time-Box, Artefatti e Regole.
I  Scrum Team sono  designati per  ottimizzare la  flessibilit\`a e  la produttivit\`a. Proprio per questo
sono auto-organizzati, cross-funzionali e lavorano su iterazioni.
Ogni Scrum Team ha tre ruoli: 

\begin{enumerate}
\item Scrum Master: si assicura che il processo \`e compreso e seguito;
\item Product Owner: responsabile di massimizzare il valore
del lavoro  che il  Team Scrum  fa; 
\item Team: fa  il lavoro. Il Team di
sviluppo ha  tutte le  competenze necessarie  per rispondere  alle esigenze  del
Product Owner e per  produrre, entro  la fine  dello Sprint, una
porzione potenzialmente rilasciabile di prodotto.
\end{enumerate}

Scrum impiega intervalli  di tempo per creare regolarit\`a.  Gli elementi che appartengono a questo intervallo temporale sono:
\begin{itemize}
\item Release  Planning  Meeting 
\item Sprint Planning  Meeting
\item Sprint
\item Daily Scrum Meeting
\item Sprint Review
\item Sprint Retrospective
\end{itemize}

Il cuore di Scrum \`e lo \textbf{Sprint}, cio\`e una iterazione di un mese o meno, coerente per
tutta la lunghezza del progetto.

Tutti gli sprint utilizzano  lo stesso framework Scrum  e tutti
gli sprint  forniscono un  incremento del  prodotto finale  che \`e potenzialmente
rilasciabile.  Uno  Sprint inizia  immediatamente  dopo l'altro.

Scrum  impiega principalmente quattro artefatti:\begin{itemize}
\item[-] Product Backlog: \`e l'elenco prioritario di tutto ci\`o che potrebbe essere necessario al prodotto;
\item[-] Sprint Backlog: \`e la lista di compiti per trasformare  l'arretrato di prodotto per uno Sprint in
un incremento  potenzialmente rilasciabile;
\item[-] Burndown: \`e la misura del Product Backlog residuo nel corso del tempo;
\item[-] Release Burndown: \`e la misura del Product Backlog rimanente  al tempo del piano di rilascio;
\item[-] Sprint Burndown: \`e la  misura degli elementi restanti dello Sprint Backlog
nel tempo  di uno  Sprint.
\end{itemize}

Le regole legano insieme i time-boxes, i ruoli e gli artefatti e vengono descritte in questo documento.
Per  esempio, \`e regola  che solo  i membri  del Team  Scrum -  le persone
impegnate a  trasformare l'arretrato  del prodotto  in un  incremento -  possono
parlare durante un Daily Scrum Meeting. Daremo dei ''TIP'' - Suggerimenti per descrivere
le modalit\`a di implementazione di  Scrum che non sono regole.
\vspace{0.4cm}

\fbox{
\begin{minipage}{0.9\textwidth}
\begin{large}TIP\end{large}\\
Quando le regole non sono precise, chi usa Scrum si aspetta di capire cosa fare. Non \`e importante cercare una soluzione perfetta, perch\`e il problema di solito cambia rapidamente. Al contrario, bisogna provare qualcosa e capirne il funzionamento. I meccanismi di natura empirica di ispezione-adeguamento di Scrum ci guideranno.
\end{minipage}
}