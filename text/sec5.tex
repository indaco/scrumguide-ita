%%%%%%%%%%%%%%%%%%%%%%%%%%
% SECTION 5 : Time-Boxes %
%%%%%%%%%%%%%%%%%%%%%%%%%%
\section*{\color{Blue}{TIME-BOXES}}
\label{sec:timeboxes}
Le Time-Boxes in Scrum sono il Release Planning Meeting, lo Sprint, lo Sprint Planning Meeting, lo Sprint Review, lo
Sprint Retrospective e il Daily Scrum.

\subsection*{\color{Blue}{RELEASE PLANNING MEETING}}
\label{sec:releaseplanningmeeting}
Lo scopo del Release Planning \`e quello di stabilire un piano e gli obiettivi che i team di Scrum e il resto delle
organizzazioni devono esser in grado di comprendere e comunicare. Il Release Planning  risponde alle domande: ''Qual \`e
il migliore modo possibile per trasformare la nostra visione in un prodotto vincente? Come possiamo raggiungere o superare la
soddisfazione desiderata del cliente e massimizzare il ritorno sull'investimento (ROI)?'' Il piano di rilascio stabilisce l'obiettivo
del rilascio, le priorit\`a pi\`u alte del Product Backlog, i rischi maggiori, le caratteristiche generali e le
funzionalit\`a che la versione da rilasciare conterr\`a. Stabilisce anche una probabile data di consegna e il costo che
dovrebbe contenere, se non cambia nulla. L'organizzazione pu\`o quindi controllare i progressi ed apportare modifiche a
questo piano Sprint-per-Sprint.
\newline

Il Release Planning Meeting \`e del tutto facoltativo. Se i Team Scrum iniziano i lavori senza la riunione,
l'assenza dei suoi artefatti sar\`a vista come un ostacolo da risolvere. Il lavoro per risolvere l'impedimento
diventer\`a un elemento del Product Backlog.
\newline

Con Scrum i prodotti sono costruiti in modo iterativo, in cui ogni Sprint crea un incremento del prodotto, a partire
dal pi\`u importante e pi\`u rischioso. Pi\`u Sprint creano pi\`u incrementi di prodotto. Ogni incremento \`e
potenzialmente rilasciabile. Quando ci sono abbastanza incrementi che danno valore al prodotto, questo viene
rilasciato. Molte organizzazioni hanno gi\`a un processo di pianificazione e di rilascio. Nella maggior parte di questi
processi la pianificazione \`e fatta all'inizio del rilascio per poi esser lasciata invariata con il passare del tempo.
In Scrum con il Release Planning Meeting vengono definiti l'obiettivo globale e i probabili esiti. Richiede solitamente non
pi\`u del 15-20\% del tempo necessario ad un'organizzazione per costruire un piano tradizionale rilascio. Tuttavia, un
rilascio gestito tramite Scrum realizza la pianificazione just-in-time ad ogni incontro di Sprint Review e di Sprint Planning ,
cos\`i come la pianificazione just-in-time quotidiana ad ogni riunione Daily Scrum. In generale, gli sforzi di rilascio
in Scrum probabilmente consumano un po' pi\`u di sforzo rispetto ad un piano tradizionale di rilascio.
\newline

Il Release Planning Meeting richiede la stima e la prioritizzazione del Product Backlog per il rilascio. Ci sono molte tecniche per
fare ci\`o, che si trovano al di fuori del terreno di competenza di Scrum, ma che sono comunque utili se usati con esso.

\subsection*{\color{Blue}{SPRINT}}
\label{sec:sprint}
Uno Sprint \`e una iterazione. Gli Sprint sono intervalli temporali. Durante lo Sprint, lo Scrum Master assicura che
non ci siano delle modifiche che alterino l'obiettivo di Sprint. Sia la composizione dei team che gli obiettivi di
qualit\`a rimangono costanti per tutta la sua durarta. Gli Sprint contengono e consistono della riunione Sprint
Planning, del lavoro di sviluppo, della Sprint Review e dello Sprint Retrospective. Gli Sprint si verificano uno dopo
l'altro, senza pause temporali tra essi.

\tip{Se un Team si accorge che si \`e impegnato a consegnare pi\`u di quanto pu\`o fare, il Team si incontra con il Product Owner per rimuovere o ridurre gli obiettivi del Product Backlog associati a quello Sprint. Se il Team crede invece di avere del tempo extra, pu\`o accordarsi con il Product Owner per selezionare altri elementi dal Product Backlog.}

Un progetto viene utilizzato per realizzare qualcosa; nel caso di sviluppo di software, \`e usato per costruire un
prodotto o un sistema. Ogni progetto consiste in una definizione di ci\`o che si va a costruire, un piano di
costruzione, il lavoro svolto in base al piano e il prodotto risultante. Ogni progetto ha un orizzonte temporale, vale
a dire i tempi per i quali il piano \`e da considerarsi buono. Se l'orizzonte \`e troppo lungo, la definizione potrebbe
essere cambiata, troppe variabili potrebbero essere entrate in gioco, il rischio pu\`o essere troppo grande ecc. Scrum \`e framework
per progetti il cui orizzonte non \`e pi\`u lungo di un mese, dove c'\`e complessit\`a sufficiente a far si che un
orizzonte temporale pi\`u lungo diventi troppo rischioso. La prevedibilit\`a del progetto deve essere controllata
almeno ogni mese, cos\`i il rischio che il progetto diventi imprevedibile e incontrollabile \`e contenuto ogni mese.

\tip{Quando un Team inizia con Scrum, Sprint di due settimane gli permettono di imparare senza brancolare nell'incertezza. Sprint di questa lunghezza possono essere sincronizzati con quelli di altri Team sommando due incrementi.}

Gli Sprint possono essere cancellati prima che la finestra di tempo dello Sprint si concluda. Solo il Product Owner ha
la facolt\`a di annullare la Sprint, anche se lui o lei pu\`o farlo anche sotto l'influenza degli stakeholder, del Team
o dello Scrum Master. Quali sono le circostanze per cui uno Sprint pu\`o essere cancellato? Potrebbe essere necessario
cancellare uno Sprint se l'obiettivo diventa obsoleto. In questo caso si tratta di una questione gestionale. Ci\`o
potrebbe verificarsi se la societ\`a cambia direzione o se cambiano le condizioni di mercato o della tecnologia. In
generale, uno Sprint deve essere annullato se, date le circostanze, non ha pi\`u senso. Tuttavia, a causa della breve
durata dell Sprint, raramente ha senso farlo.
\newline

Quando uno Sprint viene annullato, ogni elemento del Product Backlog completato e ''fatto'' viene esaminato. Sono
accettati solo gli elementi che rappresentano un incremento potenzialmente rilasciabile. Tutti gli altri elementi del Backlog
sono reimmessi nel Product Backlog con le loro stime iniziali. Qualsiasi lavoro svolto su di essi viene considerato
perduto. Terminare uno Sprint richiede consumo di risorse, dal momento che tutti vanno coinvolti in un altro incontro di
pianificazione affinch\`e si possa avviare un altro Sprint. Gli annullamenti di Sprint sono spesso traumatici per il
team, e sono molto rari.

\subsection*{\color{Blue}{SPRINT PLANNING MEETING}}
\label{sec:sprintplannnigmeeting}
Lo Sprint Planning Meeting \`e il momento in cui viene pianificata l'iterazione. Per uno Sprint della durata di un mese, questo
incontro ha la durata di otto ore. Per Sprint brevi, si pu\`o allocare circa il 5\% della lunghezza totale dello Sprint a questa
riunione. Lo Sprint Planning Meeting si compone di due parti. Nella prima parte, un tempo di quattro ore,  ci\`o che sar\`a fatto nel
Sprint \`e deciso. Nella seconda parte, le restanti quattro ore,  il Team parla di come andr\`a ad implementare
la funzionalit\`a tali da formare un incremento del prodotto durante lo Sprint.
\newline

Ci sono due parti dello Sprint Planning Meeting: ''Cosa?'' e ''Come?''. Alcuni Team Scrum combinano le due cose. Nella
prima parte, il Team Scrum affronta la questione del ''Cosa?'' Qui, il Product Owner presenta al team la parte del Product
Backlog con priorit\`a pi\`u alta. Product Owner e Team lavorano insieme per capire quali funzionalit\`a sviluppare nel corso del prossimo Sprint.
Gli input di questo incontro sono il Product Backlog, l'incremento pi\`u recente di prodotto, la capacit\`a del team ed
il suo rendimento passato. Solo il team in grado di valutare ci\`o che pu\`o compiere durante il prossimo Sprint.
\newline

Dopo aver selezionato  gli elementi dal Product Backlog, si decide l'obiettivo dello Sprint, che sar\`a raggiunto tramite
l'implementazione del Product Backlog. Questa \`e una dichiarazione che fornisce una bussola per il team che ha sempre
sotto occhio il perch\'e produrre un nuovo incremento. Lo Sprint Goal \`e un sottoinsieme dell'obiettivo di rilascio.
\newline

La ragione per avere un obiettivo di Sprint \`e quello di dare al team qualche spazio interpretativo per quanto
riguarda la funzionalit\`a. Ad esempio, l'obiettivo per lo Sprint di cui sopra potrebbe anche essere: ''Automatizzare
le funzionalit\`a di modifica dell'account cliente attraverso una possibilit\`a sicura e recuperabile di transazione a livello middleware''. Mentre il team lavora, mantiene questo obiettivo in mente. Al fine di soddisfare l'obiettivo, implementa
la funzionalit\`a e la tecnologia. Se il lavoro risulta essere pi\`u difficile di quanto il team si aspettava,
collabora con il Product Owner ed implementa solo parzialmente la funzionalit\`a. 
\newline

Nella seconda parte dello Sprint Planning Meeting, il Team affronta la questione del ''Come?'' Durante le seconde
quattro ore dallo Sprint Planning Meeting, il team calcola come trasformer\`a il Product Backlog selezionato durante
Sprint Planning Meeting (Cosa?) in un incremento di fatto. Il Team di solito inizia con la progettazione del lavoro.
Durante la progettazione, il team individua compiti. Questi compiti rappresentano i pezzi dettagliati dei lavori
necessari per convertire il Product Backlog in software. I compiti dovrebbero esser decomposti in modo tale da risultare
fattibili in meno di un giorno. Questo elenco si chiama Sprint Backlog. Il Team si auto-organizza e si impegna ad
assegnare i lavori dello Sprint Backlog, sia nel corso dello Sprint Planning Meeting sia durante lo Sprint. 
\newline

Il Product Owner \`e presente durante la seconda parte della riunione di Sprint Planning per chiarire il Product
Backlog e per contribuire a raggiungere dei compromessi. Se il Team valuta che ha troppo o troppo poco lavoro, si
pu\`o rinegoziare il Product Backlogcon il Product Owner. Il Team pu\`o anche invitare altre persone a partecipare al
fine di fornire una consulenza tecnica o su uno specifico dominio. Un nuovo team, durante questo incontro, spesso si rende subito
conto di dover affrontare il tutto come una squadra e non come unit\`a singola. Il Team si rende conto che deve fare
affidamento su se stesso. Quando si rende conto di ci\`o, comincia ad auto-organizzarsi per assumere le caratteristiche e il
comportamento di una vera squadra.

\tip{Di solito, solo il 60-70\% del Backlog totale dello Sprint viene strutturato nella riunione di Sprint Planning. La parte restante viene lasciata per essere dettagliata successivamente, o ne vengono fornite stime generiche, che verranno specificate meglio durante lo Sprint.}

\subsection*{\color{Blue}{SPRINT REVIEW}}
\label{sec:sprintreview}
Alla fine della Sprint si tiene l'incontro di Sprint Review. Quattro ore di riunione per uno Sprint di un mese. Nel
caso di Sprint di durata inferiore l'incontro non deve consumare pi\`u del 5\% del totale dello Sprint. Durante la Sprint Review, il
team di Scrum e le parti interessate discutono di ci\`o che \`e stato appena fatto. Sulla base di ci\`o, e sulle modifiche fatte al Product Backlog durante lo Sprint, si collabora per delineare cosa dovr\`a esser fatto il prossimo Sprint. Si
tratta di un incontro informale, con la presentazione della funzionalit\`a, destinato a promuovere la collaborazione su
cosa fare dopo. 
\newline

L'incontro comprende almeno i seguenti elementi. Il Product Owner identifica ci\`o che \`e stato fatto e ci\`o che non
\`e stato fatto. Il Team discute cosa \`e andato bene durante lo Sprint, quali problemi ha incontrato e come
risolverli. In pi\`u mostra il lavoro e risponde alle domande. Il Product Owner discute poi il Product
Backlog nella sua forma attuale. Lui o lei propone delle date di completamento con ipotesi di velocit\`a diverse.
L'intero gruppo collabora poi su ci\`o che ha visto e ci\`o che questo significa per le cose da fare dopo. La Sprint
Review fornisce un prezioso contributo alla successiva riunione di Sprint Planning.

\subsection*{\color{Blue}{SPRINT RETROSPECTIVE}}
\label{sec:sprintretrospective}
Dopo la Sprint Review e prima del prossimo incontro di Sprint Planning, il Team Scrum si riunisce per lo Sprint
Retrospective. In questa tre ore di riunione lo Scrum Master incoraggia il Team a rivedere, nel quadro del framework e
delle pratiche Scrum, il processo di sviluppo adottato al fine di rendere pi\`u efficace e piacevole il prossimo Sprint.
Molti libri documentano tecniche utili da usare durante la Sprint Review.
\newline

L'obiettivo della retrospezione (Sprint Review) \`e quello di esaminare come l'ultimo Sprint \`e andato per quanto riguarda le persone,
le relazioni, i processi e gli strumenti. L'ispezione dovrebbe individuare le priorit\`a e gli elementi principali che
sono andati bene e quegli elementi che, se affrontati in modo diverso, potrebbero rendere le cose migliori. Questi includono la
composizione del gruppo, le modalit\`a di riunione, gli strumenti, la definizione di ''fatto'', i metodi di
comunicazione, e i processi per trasformare gli elementi del Product Backlog in qualcosa di ''fatto.'' Entro la fine
dello Sprint Retrospective, il Team Scrum dovrebbe avere individuato le misure di miglioramenteo da attuare nei
prossimi Sprint. Questi cambiamenti diventano l'adattamento all'ispezione empirica.


\subsection*{\color{Blue}{DAILY SCRUM}}
\label{sec:dailyscrum}
Ogni team si incontra tutti i giorni per 15 minuti nel Daily Scrum. \`E un incontro che avviene sempre alla stessa ora
e sempre nello stesso luogo in cui ogni membro del team, spiega:
\begin{enumerate}
	\item ci\`o che  ha compiuto dopo l'ultima riunione;
	\item ci\`o che lui o lei si prepara a fare prima della prossima riunione; e
	\item quali ostacoli ci sono sulla sua strada.
\end{enumerate}

Il Daily Scrum migliora le comunicazioni, elimina le altre riunioni, individua e rimuove gli ostacoli allo sviluppo,
evidenzia e promuove un processo decisionale rapido ed accresce il livello di conoscenza del progetto di ciascuno.
\newline

Lo Scrum Master assicura che il team tanga la riunione. Il Team \`e responsabile della conduzione del Daily Scrum. Lo
Scrum Master insegna alla squadra a mantenere il Daily Scrum breve facendo rispettare le regole e facendo in modo che
la gente parli brevemente. Lo Scrum Master impone anche la regola che i polli non sono autorizzati a parlare n\`e a interferire in alcun
modo con il Daily Scrum.
 \newline

Il Daily Scrum non \`e una riunione di stato. Non \`e per chiunque, ma per chi pu\`o trasformare gli elementi del
Product Backlog in un incremento (il Team). Il Team si \`e impegnato per il raggiungimento dell'obiettivo di Sprint,  e per
gl elementi del Product Backlog scelti. Il Daily Scrum serve come controllo dei progressi che portano verso lo Sprint Goal (le
tre domande). 
Incontri successivi possono servire per fare adattamenti in relazione al lavoro previsto come prossimo all'interno dello Sprint.
L'intento \`e quello di ottimizzare la probabilit\`a che il team raggiunga il suo obiettivo. Nel processo
empirico alla base di Scrum il Daily Scrum rappresenta un incontro chiave di ispezione e adattamento.

