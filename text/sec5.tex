%%%%%%%%%%%%%%%%%%%%%%%%%%
% SECTION 5 : Time-Boxes %
%%%%%%%%%%%%%%%%%%%%%%%%%%
\section*{\color{Blue}{TIME-BOXES}}
\label{sec:timeboxes}
Le Time-Boxes in Scrum sono il Release Planning Meeting, lo Sprint, lo Sprint Planning Meeting, lo Sprint Review, lo Sprint Retrospective
e il Daily Scrum.

\subsection*{\color{Blue}{RELEASE PLANNING MEETING}}
\label{sec:releaseplanningmeeting}
Lo scopo del Release Planning \`e quello di stabilire un piano e gli obiettivi che i team di Scrum e il resto delle organizzazioni devono
esser in grado di comprendere e comunicare. Il Planning Release risponde alle domande: ''Qual \`e il migliore modo possibile per
trasformare la visione in un prodotto vincente? Come possiamo raggiungere o superare la soddisfazione desiderata del cliente ed avere
ritorno sull'investimento?'' Il piano di rilascio stabilisce l'obiettivo del rilascio, la priorit\`a pi\`u alta del Product Backlog, i
rischi maggiori, le caratteristiche generali e le funzionalit\`a che la versione da rilasciare conterr\`a. Stabilisce anche una probabile
data di consegna e il costo che dovrebbe contenere, se non cambia nulla. L'organizzazione pu\`o quindi controllare i progressi ed
apportare modifiche a questo piano Sprint-per-Sprint.
\newline

La pianificazione del rilascio \`e del tutto facoltativa. Se i Team Scrum iniziano i lavori senza la riunione, l'assenza dei suoi
manufatti sar\`a vista come un ostacolo da risolvere. Il lavoro per risolvere l'impedimento diventer\`a un elemento del Product Backlog.
\newline

Con Scrum i prodotti sono costruiti in modo iterativo, in cui ogni Sprint crea un incremento del prodotto, a partire dalla pi\`u
importante e pi\`u rischiosa. Pi\`u Sprint creano incrementi di prodotto. Ogni incremento \`e potenzialmente rilasciabile. Quando ci sono
abbastanza incrementi che danno valore al prodotto, questo viene rilasciato. Molte organizzazioni hanno gi\`a un processo di
pianificazione e di rilascio. Nella maggior parte di questi processi la pianificazione \`e fatta all'inizio del rilascio per poi esser
lasciato invariato con il passare del tempo. In Scrum con il Planning Release vengono definiti l'obiettivo globale e i probabili esiti.
Richiede solitamente non pi\`u del 15-20\% del tempo necessario ad un'organizzazione per costruire un piano tradizionale rilascio.
Tuttavia, un rilascio di Scrum realizza la progettazione just-in-time ad ogni incontro di Sprint Review e di Sprint Planning , cos\`i
come la pianificazione just-in-time quotidiana ad ogni riunione Daily Scrum. In generale, gli sforzi di rilascio in Scrum probabilmente
consumano un po' pi\`u di sforzo rispetto ad un piano tradizionali di rilascio.
\newline

Il Release Planning richiede la stima e la priorit\`a del Product Backlog per il rilascio. Ci sono molte tecniche per fare in modo che si
trovano al di fuori della competenza di Scrum, ma sono comunque utili se usati con esso.

\subsection*{\color{Blue}{SPRINT}}
\label{sec:sprint}
Uno Sprint \`e una iterazione. Gli Sprint sono intervalli temporali. Durante lo Sprint, lo Scrum Master assicura che non ci sono delle
modifiche che alterino l'obiettivo di Sprint. Sia la composizione dei team che gli obiettivi di qualit\`a rimangono costanti per tutta la
sua durarta. Gli Sprint contengono e consistono della riunione Sprint Planning, del lavoro di sviluppo, della Sprint Review e dello
Sprint Retrospective. Gli Sprint si verificano uno dopo l'altro, senza pause temporali tra essi.

\vspace{0.4cm}
\fbox{
	\begin{minipage}{0.9\textwidth}
		\begin{large}Suggerimento\end{large}\\
		{\color{Blue}{Da tradurre...}}
	\end{minipage}
}
\vspace{0.4cm}
\linebreak

Un progetto viene utilizzato per realizzare qualcosa; nel caso di sviluppo di software, \`e usato per costruire un prodotto o un sistema.
Ogni progetto consiste in una definizione di ci\`o che si va a costruire, un piano di costruzione, il lavoro svolto in base al piano e il
prodotto risultante. Ogni progetto ha un orizzonte temporale, vale a dire i tempi per i quali il piano \`e da considerarsi buono. Se
l'orizzonte \`e troppo lungo, la definizione potrebbe essere cambiata, troppe variabili sono gi\`a entrate, il rischio pu\`o essere
troppo grande ecc. Scrum \`e framework per progetti il cui orizzonte non \`e pi\`u lungo di un mese, dove c'\`e complessit\`a
sufficientee a far si che un orizzonte temporale pi\`u lungo diventi troppo rischioso. La prevedibilit\`a del progetto deve essere
controllata almeno ogni mese, cos\`i il rischio che il progetto diventi imprevedibile e incontrollabile \`e contenuto ogni mese.

\vspace{0.4cm}
\fbox{
	\begin{minipage}{0.9\textwidth}
		\begin{large}Suggerimento\end{large}\\
		{\color{Blue}{Quando un team inizia ad adoperare Scrum, Sprint da due settimane consentono di imparare senza navigare  nell'incertezza. Sprint di questa lunghezza possono essere sincronizzato con altri team aggiungendo due incrementi insieme.}}
	\end{minipage}
}
\vspace{0.4cm}
\linebreak

Gli Sprint possono essere cancellati prima che la finestra di tempo dello Sprint si concluda. Solo il Product Owner ha la facolt\`a di
annullare la Sprint, anche se lui o lei pu\`o farlo sotto l'influenza delle parti interessate, del team o dello Scrum Master. Quali sono
le circostanze per cui uno Sprint pu\`o essere cancellato? Potrebbe essere necessario cancellare uno Sprint se l'obiettivo diventa
obsoleto. In questo caso si tratta di una questione gestionale. Ci\`o potrebbe verificarsi se la societ\`a cambia direzione o se cambiano
le condizioni di mercato o della tecnologia. In generale, uno Sprint deve essere annullato se, date le circostanze, non ha pi\`u senso.
Tuttavia, a causa della breve durata dell Sprint, raramente ha senso farlo.
\newline

Quando uno Sprint viene annullata, qualsiasi elemento del Product Backlog completato e ''fatto'' viene esaminato. Sono accettati solo se
essi rappresentano un incremento potenzialmente rilasciabile. Tutti gli altri elementi del Backlog sono reimmessi nel Product Backlog con
le loro stime iniziali. Qualsiasi lavoro svolto su di essi viene considerato perduto. Terminare uno Sprint richiede consumo di risorse,
dal momento che tutto va raggruppato in un altro incontro di pianificazione affinch\`e si possa avviare un altro Sprint. Gli annullamenti
di Sprint sono spesso traumatici per il team, e sono molto rare.

\subsection*{\color{Blue}{SPRINT PLANNING MEETING}}
\label{sec:sprintplannnigmeeting}
L'incontro Sprint Planning avviene quando l'iterazione \`e prevista. Per uno Sprint della durata di un mese, questo incontro ha la durata
di otto ore. Per Sprint brevi, allocare circa il 5\% della lunghezza totale Sprint a questa riunione. Si compone di due parti. La prima
parte, un tempo di quattro ore, si ha quando ci\`o che sar\`a fatto nel Sprint \`e deciso. La seconda parte, le restanti quattro ore, \`e
quando il Team parla di come andr\`a ad implementare la funzionalit\`a tali da formare un incremento del prodotto durante lo Sprint.
\newline

Ci sono due parti dello Sprint Planning Meeting: ''Cosa?'' e ''Come?''. Alcuni Team Scrum combinano le due cose. Nella prima parte, il
Team Scrum affronta la questione del ''Cosa?'' Qui, il Product Owner presenta al team il Product Backlog con priorit\`a alta. Lavorano
insieme per capire quali funzionalit\`a sviluppare nel corso del prossimo Sprint. Gli input di questo incontro sono il Product Backlog,
l'incremento pi\`u recente di prodotto, la capacit\`a del team ed il suo rendimento passato. Solo il team in grado di valutare ci\`o che
pu\`o compiere durante il prossimo Sprint.
\newline

Dopo aver selezionato il Product Backlog, si realizza l'obiettivo dello Sprint che sar\`a raggiunto tramite l'implementazione del Product
Backlog. Questa \`e una dichiarazione che fornisce una bussola per il team che ha sempre sotto occhio il perch\'e produrre un nuovo
incremento. Lo Sprint Goal \`e un sottoinsieme dell'obiettivo di rilascio.
\newline

La ragione per avere un obiettivo di Sprint \`e quello di dare al team qualche spazio interpretativo per quanto riguarda la
funzionalit\`a. Ad esempio, l'obiettivo per lo Sprint di cui sopra potrebbe anche essere: ''automatizzare le funzionalit\`a di modifica
dell'account cliente attraverso una possibilit\`a sicura e recuperabile del middleware di transazione''. Mentre il team lavora, mantiene
questo obiettivo in mente. Al fine di soddisfare l'obiettivo, implementa la funzionalit\`a e la tecnologia. Se il lavoro risulta essere
pi\`u difficile di quanto il team si aspettava, collabora con il Product Owner ed implementa solo parzialmente la funzionalit\`a.
\newline

Nella seconda parte dello Sprint Planning Meeting, il Team affronta la questione del ''Come?'' Durante le seconde quattro ore dallo
Sprint Planning Meeting, il team calcola come trasformer\`a il Product Backlog selezionato durante Sprint Planning Meeting (Cosa?) in un
incremento di fatto. Il Team di solito inizia con la progettazione del lavoro. Durante la progettazione, il team individua compiti.
Questi compiti rappresentano i pezzi dettagliati dei lavori necessari per convertire il Product Backlog in software. Icompiti dovrebbero
esser decomposti in modo tale da risultare fattibili in meno di un giorno. Questo elenco si chiama Sprint Backlog. Il Team si
auto-organizza e si impegna ad assegnare i lavori dello Sprint Backlog, sia nel corso della riunione Sprint Planning sia durante lo
Sprint.
\newline

Il Product Owner \`e presente durante la seconda parte della riunione di Sprint Planning per chiarire il Product Backlog e per
contribuire a raggiungere dei compromessi. Se la squadra determina che ha troppo o troppo poco lavoro, si pu\`o rinegoziare il Product
Backlogcon il Product Owner. Il Team pu\`o anche invitare altre persone a partecipare al fine di fornire una consulenza tecnica o di
dominio. Un nuovo team, durante questo incontro, spesso si rende subito conto di dover affrontare il tutto come una squadra e non come
unit\`a singola. Il Team si rende conto che deve fare affidamento su se stessa. Come si realizza questo, comincia ad auto-organizzarsi
per assumere le caratteristiche e il comportamento di una vera squadra.

\vspace{0.4cm}
\fbox{
	\begin{minipage}{0.9\textwidth}
		\begin{large}Suggerimento\end{large}\\
		{\color{Blue}{Di solito, solo il 60-70\% dello Sprint Backlog totale sar\`a messo a punto nel corso della riunione Sprint Planning. Il resto viene dettagliato successivamente.}}
		\end{minipage}
}
\vspace{0.4cm}
\linebreak


\subsection*{\color{Blue}{SPRINT REVIEW}}
\label{sec:sprintreview}
Alla fine della Sprint si tiene l'incontro di Sprint Review. Quattro ore di riunione per uno Sprint di un mese. Nel caso di durata
inferiore l'incontro non deve consumare pi\`u del 5\% del totale Sprint. Durante la Sprint Review, il team di Scrum e le parti
interessate collaborano su ci\`o che \`e stato appena fatto. Sulla base di ci\`o si modifica il Product Backlog durante lo Sprint e si
collabora per delineare cosa dovr\`a esser fatto il prossimo Sprint. Si tratta di un incontro informale, con la presentazione della
funzionalit\`a destinate a promuovere la collaborazione su cosa fare dopo.
\newline

L'incontro comprende almeno i seguenti elementi. Il Product Owner identifica ci\`o che \`e stato fatto e ci\`o che non \`e stato fatto.
Il Team discute cosa \`e andato bene durante lo Sprint, quali problemi ha incontrato e come risolverlie questi. In pi\`u dimostra il
lavoro e risponde alle domande. Il Product Owner discute poi il Product Backlog nella sua forma attuale. Lui o lei propone delle date di
completamento con ipotesi di velocit\`a diverse. L'intero gruppo collabora poi su ci\`o che ha visto e ci\`o che questo significa per le
cose da fare dopo. La Sprint Review fornisce un prezioso contributo alla successiva riunione di Sprint Planning.


\subsection*{\color{Blue}{SPRINT RETROSPECTIVE}}
\label{sec:sprintretrospective}
Dopo lo Sprint Review e prima del prossimo incontro di Sprint Planning, il Team Scrum si riunisce per lo Sprint Retrospective. In questa
tre oredi riunione lo Scrum Master incoraggia il Team a rivedere, nel quadro del framework e delle pratiche Scrum, il processo di
sviluppo adottato al fine di rendere pi\`u efficace e piacevole i prossimi Sprint. Molti libri documentano tecniche di documento utili da
usare durante la retrospezione.
\newline

L'obiettivo della retrospezione \`e quello di esaminare come l'ultimo Sprint \`e andato per quanto riguarda le persone, le relazioni, i
processi e gli strumenti. L'ispezione dovrebbe individuare le priorit\`a e gli elementi principali che sono andati bene e quelle voci
che, se fatto in modo diverso, potrebbe rendere le cose migliori. Questi includono la composizione del gruppo, le modalit\`a di riunione,
gli strumenti, la definizione di ''fatto'', i metodi di comunicazione, e dei processi per rendere gli elementi del Product Backlog in
qualcosa di ''fatto.'' Entro la fine dello Sprint Retrospective, il Team Scrum dovrebbe avere individuato le misure di miglioramenteo da
attuare nei prossimi Sprint. Questi cambiamenti diventano l'adattamento al controllo empirico.


\subsection*{\color{Blue}{DAILY SCRUM}}
\label{sec:dailyscrum}
Ogni team si incontra tutti i giorni per 15 minuti nel Daily Scrum. \`E un incontro che avviene sempre alla stessa ora e sempre nello
stesso luogo in cui ogni membro del team, spiega:
\begin{enumerate}
	\item ci\`o che  ha compiuto dopo l'ultima riunione;
	\item ci\`o che lui o lei si prepara a fare prima della prossima riunione; e
	\item quali ostacoli ha incontrato lungo ilsuo cammino.
\end{enumerate}

Il Daily Scrum migliora le comunicazioni, elimina le altre riunioni, di individua e rimuove gli ostacoli allo sviluppo, evidenzia e
promuove un processo decisionale rapido ed accresce il livello individuale di conoscenza del progetto.
\newline

Lo Scrum Master assicura che il team tanga la riunione. Il Team \`e responsabile della conduzione del Daily Scrum. Lo Scrum Master
insegna alla squadra a mantenere il Daily Scrum breve facendo rispettare le regole e facendo in modo che la gente parli brevemente. Lo
Scrum Master impone anche la regola che i polli non sono autorizzati a parlare o in alcun modo ad interferire con il Daily Scrum.
\newline

Il Daily Scrum non \`e una riunione di stato. Non \`e per chiunque, ma per chi pu\`o trasformare gli elementi del Product Backlog in un
incremento (il Team). Il Team si \`e impegna per il raggiungimento dell'obiettivo di Sprint e per gl elementi del Product Backlog. Il
Daily Scrum serve come controllo dei progressi che portano verso lo Sprint Goal (le tre domande). L'intento \`e quello di ottimizzare la
probabilit\`a che il team raggiunga il suo obiettivo. Nel processo empirico alla base di Scrum rappresenta una chiave di controllo ed un
incontro di adattamento.