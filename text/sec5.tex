%%%%%%%%%%%%%%%%%%%%%%%%%%%%%%%%%%%%%%
% SECTION 5 : Gli Artefatti di Scrum %
%%%%%%%%%%%%%%%%%%%%%%%%%%%%%%%%%%%%%%

\section*{\color{Blue}{Gli Artefatti di Scrum}}
\label{sec:artifacts}
\addcontentsline{toc}{section}{Gli Artefatti di Scrum}
\pdfbookmark[16]{Gli Artefatti di Scrum}{artifacts}
Gli artefatti di Scrum rappresentano il lavoro o il valore in diversi modi tale da risultare utili a fornire trasparenza e 
opportunit\`a di ispezione e adattamento. Gli artefatti definiti da Scrum sono specificatamente progettati per massimizzare la 
trasparenza delle informazioni chiave necessarie ad assicurare ai Team Scrum il successo nella realizzazione di un Incremento 
``Fatto''.

\subsection*{\color{SteelBlue}{Product Backlog}}
\label{sec:product_backlog}
\addcontentsline{toc}{subsection}{Product Backlog}
\pdfbookmark[17]{Product Backlog}{product_backlog}
Il Product Backlog \`e un elenco ordinato di tutto ciò che potrebbe essere necessario al prodotto ed \`e l'unica fonte di 
requisiti per le modifiche da apportare al prodotto. Il Product Owner \`e il responsabile del Product Backlog, compreso il suo 
contenuto, la sua disponibilità e l'ordinamento dei suoi elementi. \newline

\\Un Product Backlog non \`e mai completo. Il suo primo sviluppo definisce solo i requisiti inizialmente conosciuti e 
meglio compresi. Il Product Backlog evolve come il prodotto e l'ambiente in cui verrà utilizzato. \`E dinamico e cambia 
continuamente per identificare ci\`o che serve al prodotto per risultare adeguato, competitivo e utile. Finch\`e esiste un 
prodotto, esiste anche un Product Backlog. \newline

\\Il Product Backlog elenca tutte le caratteristiche, le funzioni, i requisiti, le migliorie e le correzioni che costituiscono 
le modifiche al prodotto nelle future versioni. I suoi elementi hanno una descrizione, un ordine ed una stima. \newline

\\ \`E spesso ordinato per valore, per rischio, per priorità e necessità. L'ordinamento dall'alto verso il basso degli 
elementi del Product Backlog guida le attività di sviluppo. Pi\`u alto \`e l'ordine, pi\`u la voce del Product Backlog \`e 
stata considerata ed il consenso su di essa e sul suo valore \`e maggiore. \newline

\\Gli elementi con ordine superiore sono pi\`u chiari e meglio dettagliati rispetto a quelli con ordine inferiore. Le stime 
risultano pi\`u precise e si basano sulla maggiore chiarezza e maggior numero di dettagli; minore è l'ordine e minore \`e il 
livello di dettagli. Gli elementi del Product Backlog che occuperanno il Team di Sviluppo per il prossimo Sprint sono a grana 
fine, dopo essere stati suddivisi in modo tale che ogni elemento pu\`o essere considerato un ``Fatto'' entro i tempi dello 
Sprint. Gli elementi del Product Backlog che possono essere ``Fatti'' dal Team di Sviluppo durante un Sprint sono considerati 
``pronti'' o ``perseguibili'' per essere selezionati durante l'incontro di Sprint Planning. \newline

\\Quando un prodotto viene utilizzato, acquista valore e il mercato fornisce un feedback, il Product Backlog diventa un elenco 
pi\`u completo ed  esaustivo. I requisiti non smettono mai di cambiare, cos\`i il Product Backlog \`e un artefatto vivente. I 
cambiamenti nei requisiti di business, le condizioni di mercato o la tecnologia possono causare cambiamenti nel Product 
Backlog. \newline

\\Numerosi Team Scrum si trovano spesso a lavorare sullo stesso prodotto. Un Product Backlog \`e usato per descrivere il lavoro 
imminente sul prodotto. \`E poi impiegato un backlog degli attributi del prodotto che raggruppa gli elementi .\newline
\\ Il grooming del Product Backlog \`e l'atto di aggiungere dettagli, stime e ordine agli elementi del Product Backlog. Questo 
\`e un processo continuo in cui il Product Owner e il Team di Sviluppo collaborano sui dettagli degli elementi del Product 
Backlog. Durante il grooming del Product Backlog, i suoi elementi vengono riesaminati e rivisti. Tuttavia, essi possono essere 
aggiornati in qualsiasi momento dal Product Owner o a discrezione del Product Owner. \newline

\\Il grooming \`e un'attivit\`a  part-time durante uno Sprint tra il Product Owner ed il Team di Sviluppo. Spesso il Team di 
Sviluppo ha la conoscenza di dominio per eseguire il grooming stesso. \`E il Team Scrum che decide come e quando fare grooming. 
L'attivit\`a di grooming consuma solitamente non pi\`u del 10\% della capacità del Team di Sviluppo. \newline
\\Il Team di Sviluppo \`e responsabile di tutte le stime. Il Product Owner pu\`o influenzare il Team aiutando a capire e 
selezionare i trade-off, ma chi eseguer\`a il lavoro fa la stima finale.

\subsubsection*{\color{SteelBlue}{Monitorare i Progressi verso un Obiettivo}} % (fold)
\label{ssub:monitoring_progress_toward_a_great}
In qualsiasi momento il rimanente lavoro totale per raggiungere un obiettivo pu\`o essere sommato. Il 
Product Owner traccia il lavoro totale almeno ad ogni Sprint Review. Confronta questa cifra con il 
lavoro rimanente delle precedenti Sprint Review e valuta i progressi verso il completamento dei 
lavori per raggiungere l'obiettivo previsto al tempo desiderato. Queste informazioni vengono rese trasparenti a 
tutti gli stakeholder.\newline

\\Scrum non considera il tempo impiegato per lavorare sugli elementi del Product Backlog. Il lavoro 
rimanente e la data sono le uniche  variabili di interesse. \newline

\\Diversi trend burndown, burnup e altre pratiche proiettive sono state utilizzate per prevedere il 
progresso. Queste si sono rilevate utili. Tuttavia, queste tecniche non sostituiscono l'importanza 
dell'empirismo. In ambienti complessi, cosa accadr\`a \`e sconosciuto, solo ci\`o che \`e successo pu\`o 
essere utilizzato in un futuro processo decisionale.
% subsubsection monitoring_progress_toward_a_great (end)

% subsection product_backlog (end)

\subsection*{\color{SteelBlue}{Sprint Backlog}}
\label{sec:sprint_backlog}
\addcontentsline{toc}{subsection}{Sprint Backlog}
\pdfbookmark[18]{Sprint Backlog}{sprint_backlog}
Lo Sprint Backlog \`e l'insieme delle voci selezionate dal Product Backlog per lo Sprint, 
pi\`u un piano per la distribuzione dell'incremento di prodotto e la realizzazione 
dell'obiettivo di Sprint. Lo Sprint Backlog rappresenta una previsione del Team di 
Sviluppo circa la funzionalità che rappresenter\`a  l'incremento successivo ed il lavoro 
necessario per rilasciare questa funzionalit\`a. \newline

\\ Lo Sprint Backlog definisce il lavoro che il Team di Sviluppo eseguir\`a per 
trasformare gli elementi del Product Backlog in un incremento ``Fatto''. Lo Sprint Backlog rende visibile tutto il 
lavoro che il Team di Sviluppo identifica come necessario per soddisfare l'obiettivo di Sprint. \newline

\\Lo Sprint Backlog \`e un piano con dettagli sufficienti affinch\`e i cambiamenti in atto 
possono essere compresi nel Daily Scrum. Il Team di Sviluppo modifica lo Sprint Backlog 
durante tutto lo Sprint e lo Sprint Backlog emerge durante lo Sprint. Questa emergenza si 
verifica quando il Team di Sviluppo opera attraverso il piano e viene a conoscenza di 
pi\`u dettagli sul lavoro necessario per raggiungere l'obiettivo di Sprint. \newline

\\Quando del nuovo lavoro risulta necessario, il Team di Sviluppo lo aggiunge allo Sprint 
Backlog. Poich\`e il lavoro viene eseguito o completato, il rimanente lavoro stimato viene 
aggiornato. Quando gli elementi del piano sono ritenute inutili, vengono rimossi. Solo il 
Team di Sviluppo pu\`o cambiare il suo Sprint Backlog nel corso di uno Sprint. Lo Sprint 
Backlog ha una grande visibilit\`a, \`e un'immagine in tempo reale del lavoro che il Team 
di Sviluppo prevede di realizzare durante lo Sprint; appartiene soltanto al Team di Sviluppo.

\subsubsection*{\color{SteelBlue}{Monitorare i Progressi dello Sprint}} % (fold)
\label{ssub:monitoring_sprint_progress}
In qualsiasi istante temporale durante uno Sprint, il rimanente lavoro totale del Product Backlog 
pu\`o essere sintetizzato. Il Team di Sviluppo traccia il rimanente lavoro totale quantomeno 
ad ogni Daily Scrum. Traccia queste somme giornalmente e progetta la probabilit\`a di 
raggiungere l'obiettivo dello Sprint. Tracciando il lavoro rimanente per tutto lo Sprint, 
il Team di Sviluppo \`e in grado di gestire il proprio progresso. \newline

\\Scrum non considera il tempo impiegato per lavorare sugli elementi dello Sprint Backlog. 
Il lavoro rimanente e la data sono le uniche variabili di interesse. 
% subsubsection monitoring_sprint_progress (end)

% subsection sprint_backlog (end)

\subsection*{\color{SteelBlue}{Incremento}}
\label{sec:increment}
\addcontentsline{toc}{subsection}{Incremento}
\pdfbookmark[19]{Incremento}{increment}
L'incremento \`e la somma di tutti gli elementi del Product Backlog completati durante uno 
Sprint e tutti gli Sprint precedenti. Alla fine di uno Sprint, il nuovo Incremento deve 
essere ``Fatto'', il che significa che deve essere utilizzabile e deve incontrare la 
definizione di ``Fatto'' data dal Team Scrum. Deve essere utilizzabile indipendentemente dal 
fatto che il Product Owner decide di rilasciarlo realmente.
% subsection increment (end)

%% section artifacts (end)
