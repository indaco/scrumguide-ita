%%%%%%%%%%%%%%%%%%%%%%%%%%%%%%%
% SECTION 1 : Acknowledgement %
%%%%%%%%%%%%%%%%%%%%%%%%%%%%%%%
\section*{\color{Blue}{RINGRAZIAMENTI}}
\label{sec:acknowledgements}

\subsection*{\color{Blue}{GENERALI}}
\label{sec:general}
Scrum \`e basato su un insieme di prassi accettate nel settore industriale, utilizzate e collaudate da decenni. Si tratta di una teoria di processo con base empirica. Come disse Jim Coplien a Jeff: \flqq Scrum piacer\`a a chiunque; \`e ci\`o che gi\`a facciamo quando siamo con le spalle al muro.\frqq

\subsection*{\color{Blue}{PERSONE}}
\label{sec:people}
Delle migliaia di persone che hanno contribuito a Scrum, dovremmo individuare coloro che hanno contribuito nei suoi primi dieci anni. Prima c'erano Jeff Sutherland, in collaborazione con Jeff Mckenna, e Ken Schwaber con Mike Smith e Chris Martin. Scrum \`e stato formalmente presentato e pubblicato al OOPSLA del 1995. Nei 5 anni successivi, Mike Bandle e Martine Devos contribuirono in modo significativo. E per finire tutti gli altri, senza il loro aiuto non sarebbe stato possibile ridefinire Scrum in ci\`o che oggi \`e.

\subsection*{\color{Blue}{STORIA}}
\label{sec:history}
La storia di Scrum pu\`o gi\`a essere considerata lunga nel mondo dello sviluppo software.
Per ricordare i primi posti in cui \`e stato richiesto ed adottato, onoriamo Individual, Inc. Fidelity Investments, e IDX (oggi GE Medical).

%\section*{\color{Blue}{NOTE ALLA VERSIONE ITALIANA}}
%\label{sec:italianversion}
%TODO