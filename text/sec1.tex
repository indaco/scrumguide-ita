%%%%%%%%%%%%%%%%%%%%%%%%%%%%%%%%%%%%%%%%%%%%%%%%%%%%%%%%%%
% SECTION 1 : Scopo della Guida Scrum, Overview di Scrum %
%%%%%%%%%%%%%%%%%%%%%%%%%%%%%%%%%%%%%%%%%%%%%%%%%%%%%%%%%%

\section*{\color{Blue}{Scopo della Guida Scrum}}% (fold)
\label{sec:purpose}
\addcontentsline{toc}{section}{Scopo della Guida Scrum}
\pdfbookmark[1]{Scopo della Guida Scrum}{purpose}
Scrum \`e un framework per sviluppare e sostenere prodotti complessi. Questa guida contiene la definizione di Scrum. Questa 
definizione \`e costituita dai ruoli, gli eventi e gli artefatti di Scrum e le regole che li legano insieme. Ken Schwaber e Jeff 
Sutherland hanno sviluppato Scrum; la Guida Scrum \`e scritta e distribuita da loro.
% section purpose (end)

\section*{\color{Blue}{Overview di Scrum}}% (fold)
\label{sec:overview}
\addcontentsline{toc}{section}{Overview di Scrum}
\pdfbookmark[2]{Overview di Scrum}{overview}
Scrum (n): Un framework con il quale le persone possono affrontare complessi problemi di adattamento mentre in modo 
produttivo e creativo rilasciano prodotti dal pi\`u alto valore possibile. Scrum \`e:

\begin{itemize}
\item Leggero
\item Semplice da comprendere
\item Estremamente difficile da padroneggiare
\end{itemize}

Scrum \`e un framework di processo utilizzato a partire dai primi anni 1990 per gestire lo sviluppo di prodotti complessi. 
Scrum non \`e un processo o una tecnica per costruire prodotti ma piuttosto è un framework all'interno del quale \`e possibile 
utilizzare vari processi e tecniche. Scrum rende chiara l'efficacia relativa del tuo product management e delle pratiche di 
sviluppo usate in modo da poterle migliorare.

\subsection*{\color{SteelBlue}{Scrum Framework}}% (fold)
\label{sec:framework}
\addcontentsline{toc}{subsection}{Scrum Framework}
\pdfbookmark[3]{Scrum Framework}{framework}
Il framework Scrum \`e costituito dai Team Scrum e dai ruoli, eventi, artifatti e regole ad essi associati. Ogni componente del 
framewrok serve ad uno specifico scopo ed \`e essenziale per il successo e l'utilizzo di Scrum.
\newline
\\Strategie specifiche per l'utilizzo del framework Scrum variano e sono descritte altrove.
\newline
\\ Le regole di Scrum legano insieme gli eventi, i ruoli e gli artefatti governando le relazioni e le interazioni tra essi e 
sono descritte in tutto il corpo di questo documento.
% subsection framework (end)

%%section overview (end)