%%%%%%%%%%%%%%%%%%%%%%%%%%%%%%%%%%%%%%%%%%%%%%%%%%%%%%
% SECTION 7 : Conclusioni, Ringraziamenti, Revisioni %
%%%%%%%%%%%%%%%%%%%%%%%%%%%%%%%%%%%%%%%%%%%%%%%%%%%%%%

\section*{\color{Blue}{Conclusioni}}
\label{sec:conclusion}
\addcontentsline{toc}{section}{Conclusioni}
Scrum è gratuito ed offerto in questa guida. I ruoli, gli artefatti, gli eventi e le regole di Scrum sono immutabili e anche se \`e 
possibile implementare solo alcune parti di Scrum, il risultato non \`e Scrum. Scrum esiste solo nella sua interezza e funziona bene come 
contenitore per altre tecniche, metodologie e pratiche.


\section*{\color{Blue}{Ringraziamenti}}
\label{sec:acknowledgements}
\addcontentsline{toc}{section}{Ringraziamenti}

\subsection*{\color{SteelBlue}{Persone}}
\label{sec:people}
\addcontentsline{toc}{subsection}{Persone}
Delle migliaia di persone che hanno contribuito a Scrum, dovremmo individuare coloro che hanno contribuito nei suoi
primi dieci anni. Prima c'erano Jeff Sutherland, in collaborazione con Jeff McKenna, e Ken Schwaber con Mike Smith e
Chris Martin. Molti altri hanno contribuito negli anni successivi e senza il loro aiuto non sarebbe stato possibile ridefinire Scrum in 
ci\`o che oggi \`e. David Starr ha approfondito e fornito le competenze editoriali per la formulazione di questa versione della Guida Scrum.

\subsection*{\color{SteelBlue}{Storia}}
\label{sec:history}
\addcontentsline{toc}{subsection}{Storia}
Ken Schwaber e Jeff Sutherland hanno presentato per la prima volta Scrum alla conferenza OOPSLA del 1995. Questa presentazione ha 
essenzialemente documentato ci\`o che Ken e Jeff avevano appreso negli anni precedenti applicando Scrum.
\newline
\\La storia di Scrum pu\`o gi\`a essere considerata lunga. Per ricordare i primi posti in cui \`e stato richiesto ed adottato, onoriamo 
Individual, Inc. Fidelity Investments, e IDX (oggi GE Medical).

\newpage
\section*{\color{Blue}{Revisioni}}
\label{sec:revisions}
Questa versione di luglio 2011 della Guida Scrum è diversa dalla precedente di febbraio 2010. In particolare abbiamo cercato di rimuovere le 
tecniche o le best practices dal nucleo di Scrum. Questi variano a seconda della circostanza. Inizieremo pi\`u tardi con un compedio di 
``Best Practices'' per offrire alcune delle nostre esperienze.
\newline
\\La Guida Scrum documenta come Scrum \`e stato sviluppato e sostenuto da pi\`u di venti anni da Jeff Sutherland e Ken Schwaber. Altre fonti 
forniscono modelli, processi e idee sulle pratiche, le agevolazioni e gli strumenti che completano il framewok Scrum. Queste ottimizzano la 
produttività, il valore, la creatività e l'orgoglio.
\newline
\\Le note di rilascio che coprono le seguenti differenze tra questa e la versione di Febbraio 2010 saranno pubblicate altrove, comprese le 
discussioni su:

\begin{enumerate}
 	\item Release Planning
	\item Release Burndown
	\item Sprint Backlog
	\item Product e Sprint Backlog Burndown
	\item Il commit \`e ora previsto
	\item Team (per Development Team)
	\item Maiali e Polli...la tradizione di Scrum
	\item Ordinati invece che prioritizzati 
\end{enumerate}

\newpage
\section*{\color{Blue}{Traduzione}}
\label{sec:translation}
Questo documento \`e stato tradotto dalla versione originale inglese di Ken Schwaber e Jeff Sutherland. Hanno contribuito a realizzare la 
traduzione Carlo Beschi e Mirco Veltri.

\subsection*{\color{SteelBlue}{Note alla versione italiana}}
\label{sec:transnotes}
I sorgenti della traduzione sono disponibili sul repository \htmladdnormallink{GitHub}{http://github.com/indaco/scrumguide-ita}. 
Sentitevi liberi di contattare i traduttori per segnalare errori o semplicemente per lasciare un feedback. Grazie.