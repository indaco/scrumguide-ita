%%%%%%%%%%%%%%%%%%%%%%%%%%%%%%
% SECTION 7 : Final Thoughts %
%%%%%%%%%%%%%%%%%%%%%%%%%%%%%%
\newpage
\section*{\color{Blue}{CONSIDERAZIONI FINALI}}
\label{sec:finalthoughts}
Alcune organizzazioni non sono in grado di costruire un incremento completo del prodotto all'interno di uno Sprint. Non possono ancora avere l'infrastruttura automatizzata di test per completare tutti i test. In questo caso, due categorie di controllo vengono create per ogni incremento: il lavoro ``fatto'' e quello ``non-fatto''. Il ``non-fatto'' \`e la parte di ciascun incremento che dovr\`a essere completato in un secondo momento. Il Product Owner sa esattamente ci\`o che lui o lei controlla alla fine della Sprint perch\`e l'incremento soddisfi la definizione di ``fatto'' e ne comprende la definizione. Il ''non-fatto'' viene aggiunto come elemento al Product Backlog e lo si chiama ''lavoro non-fatto'', cos\`i si accumula e viene riflesso correttamente sul grafico Release Burndown. Questa tecnica crea trasparenza nel progresso. La visione e l'adeguamento della Sprint Review sono accurati quanto la trasparenza.\\

Ad esempio, se un Team non \`e in grado di fare performance, regressione, stabilit\`a, sicurezza e test di integrazione per ogni voce di Product Backlog, la percentuale di questo lavoro che pu\`o essere fatto (analisi, design, refactoring, programmazione, documentazione , testing di unit\`a e di utente) \`e calcolata. Diciamo che questa proporzione \`e di sei pezzi di ``fatto'' e quattro di ``annullato''. Se il Team termina un elemento di Product Backlog di sei unit\`a di lavoro (il team fa la stima basandosi su ci\`o che sa ``fare''), quattro \`e aggiunto al ``lavoro'' non-fatto alla voce Product Backlog quando sono conclusi.\\

Sprint dopo Sprint, il ``non-fatto'' di ogni incremento \`e accumulato e deve essere affrontato prima di un rilascio. Questo lavoro si accumula linearmente anche se \`e in realt\`a c'\`e una sorta di accumulazione esponenziale che dipende dalle caratteristiche di ciascuna organizzazione. I Release Sprint sono aggiunti alla fine di ogni release per completare il lavoro ``non-fatto''.
Il numero di sprint \`e imprevedibile nella misura in cui l'accumulo di lavoro ``non-fatto'' non risulta lineare. 