%%%%%%%%%%%%%%%%%%%%%%%%%%%%%%
% SECTION 7 : Final Thoughts %
%%%%%%%%%%%%%%%%%%%%%%%%%%%%%%
\newpage
\section*{\color{Blue}{CONSIDERAZIONI FINALI}}
\label{sec:finalthoughts}
Alcune organizzazioni non sono in grado di costruire un incremento completo all'interno di un'unico Sprint. Potrebbero non avere ancora l'infrastruttura  di test automatizzati per completare tutti i test. In questo caso,  vengono create due categorie di controllo per ogni incremento: il lavoro ``fatto'' e quello ``non-fatto''. Il ``non-fatto'' \`e la parte di ciascun incremento che dovr\`a essere completato in un secondo momento. Il Product Owner sa esattamente ci\`o che lui sta ispezionando alla fine di uno Sprint, perch\`e l'incremento soddisfa la definizione di ``fatto'' e lui ne comprende la definizione. Il lavoro ''non-fatto'' viene aggiunto a un elemento del Product Backlog chiamato ''lavoro non-fatto'', e in questo modo \`e possibile accumularlo, e lo stesso viene riflesso correttamente sul grafico del Release Burndown. Questa tecnica crea trasparenza rispetto al progresso verso una release. L'ispezione e adattamento della Sprint Review sono tanto accurati quanto lo \`e questa trasparenza.\\

Ad esempio, se un Team non \`e in grado di fare test di performance, di regressione, di stabilit\`a, di sicurezza e di integrazione per ogni elemento del Product Backlog, viene calcolata la proporzione di questo lavoro rispetto a quello che pu\`o essere fatto (analisi, design, refactoring, programmazione, documentazione , unit tests e test a livello utente). Mettiamo che la proporzione sia di sei parti di ``fatto'' e quattro di ``non fatto''. Se il Team completa un elemento del Product Backlog corrispondente a sei unit\`a di lavoro (il team fa le sue stime basandosi sulle cose che sa  come ``fare''), un quattro viene aggiunto alla voce ``lavoro non-fatto'' del Product Backlog.\\

Sprint dopo Sprint, il lavoro ``non-fatto'' relativo a  ogni incremento viene accumulato, e deve essere affrontato prima del rilascio del prodotto. Questo lavoro si accumula linearmente, anche se di fatto vi \`e una sorta di accumulazione esponenziale, che dipende dalle caratteristiche di ciascuna organizzazione. I Release Sprint vengono aggiunti nella parte finale di ogni release proprio per completare il lavoro ``non-fatto''.
Il numero di sprint necessari resta imprevedibile nella misura in cui l'accumulo di lavoro ``non-fatto'' non \`e  lineare. 
